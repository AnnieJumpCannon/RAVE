%% This file is part of the UNRAVE Project 
%% Copyright 2016 the authors.  All rights reserved.


%% TODO:
%% Co-authors: Look for \stub{}s. If you can expand it, do so!

\documentclass[preprint,trackchanges]{aastex}

\usepackage{amsmath}
\usepackage{bm}
\usepackage{xcolor}

\definecolor{unoffensive-warning}{HTML}{B4DCED}

% This section can be removed at submission.
% ------------------------------------------
%\usepackage{color}
\usepackage{datenumber}

\newcounter{dateone}
\newcounter{datetwo}

\newcommand{\difftoday}[3]{%
  \setmydatenumber{dateone}{\the\year}{\the\month}{\the\day}%
  \setmydatenumber{datetwo}{#1}{#2}{#3}%
  \addtocounter{datetwo}{-\thedateone}%
  \the\numexpr(\thedatetwo)\relax\space days
}
% ------------------------------------------

\IfFileExists{vc.tex}{\input{vc.tex}}{\newcommand{\githash}{UNKNOWN}\newcommand{\giturl}{UNKNOWN}}

\newcommand{\acronym}[1]{{\small{#1}}}

\newcommand{\project}[1]{\textsl{#1}}
\newcommand{\gaia}{\project{Gaia}}
\newcommand{\thecannon}{\project{The~Cannon}}
\newcommand{\rave}{\project{\acronym{RAVE}}}
\newcommand{\galah}{\project{\acronym{GALAH}}}
\newcommand{\ges}{\project{Gaia-ESO}}
\newcommand{\apogee}{\project{\acronym{APOGEE}}}
\newcommand{\aspcap}{\project{\acronym{ASPCAP}}}
\newcommand{\lamost}{\project{\acronym{LAMOST}}}
\newcommand{\hipparcos}{\project{Hipparcos}}
\newcommand{\epic}{\project{K2/EPIC}}
\newcommand{\sdss}{\project{\acronym{SDSS}}}
\newcommand{\tgas}{\project{\acronym{TGAS}}}
\newcommand{\unrave}{\project{unRAVE}}

\newcommand{\stub}[1]{{\color{blue} \textbf{#1}}}

\newcommand{\teff}{T_{\mathrm{eff}}}
\newcommand{\logg}{\log g}
\newcommand{\feh}{[\mathrm{Fe/H}]}

\newcommand{\Nspectra}{520,782}
\newcommand{\Nstars}{457,589}
\newcommand{\Nstarsqc}{434,470}


\newcommand{\Dvector}[1]{\boldsymbol{#1}}
\newcommand{\vectheta}{\Dvector{\theta}}
\newcommand{\vecv}{\Dvector{v}}
\newcommand{\argmin}[1]{\underset{#1}{\operatorname{argmin}}\,}


% For AASTeX v6
%\AuthorCallLimit=10
%\fullcollaborationName{The \rave\ Collaboration}

\begin{document}
% Remove at submission:
\slugcomment{{\color{red} \textbf{To appear on arXiv on 13 September 2016 (\difftoday{2016}{9}{9}away)}}}


\title{The \project{unRAVE} catalog}

\author{Andrew R. Casey}
\affil{Institute of Astronomy, Madingley Road, Cambridge CB3 0HA}

% Specific contributors to this work:
% There is some order already in mind, but final order will be determined by  contribution.
\author{Some combination of: Harry Enke, Gerry Gilmore, Keith Hawkins, David W. Hogg, Georges Kordopatis, Gal Matijevic, Melissa Ness, Jason Sanders, Matthias Steinmetz, Hans Walter-Rix}

% RAVE DR5 core team:
\author{Luca Casagrande, Cristina Chiappini, Andrea Kunder, Paul McMillan, Alessandro Siviero, Marica Valentini, Jennifer Wojno, Toma{\^z} Zwitter}

\author{and the \rave\ collaboration}

\begin{abstract}
The Milky Way is a powerful laboratory for understanding galaxy formation,
as it can provide orbits, ages, physical parameters, as well as chemical 
abundances for vast sets of individual stars.  These inferences require 
both astrometric and spectroscopic data.  Indeed, the Tycho-Gaia 
Astrometric Solution (\tgas) sample in the first \gaia\ data release 
requires a spectroscopic complement that includes radial velocities, 
stellar parameters, and elemental abundances.  Among existing spectroscopic 
samples, the RAdial Velocity Experiment (\rave) survey has the largest 
overlap with \tgas: 292,036 stars.  Here we present a data-driven 
re-analysis of \rave\ spectra, using an implementation of \thecannon, that
yields more precise and accurate stellar parameters and abundances.  We 
derive, and validate, improved effective temperature $\teff$, surface 
gravity $\logg$, and chemical abundances of up to seven elements (O, Mg, 
Al, Si, Ca, Fe, Ni) for \Nstarsqc\ stars.  The typical precision in chemical 
abundances is X.XX~dex for stars on the giant branch.  We provide 
spectrophotometric distances based on our stellar parameters, which should 
be superior to the parallax-based distances for faint or distant objects in
\tgas.  The \unrave\ catalog presented here, in conjunction with \tgas, will 
be the most powerful data set for chemo-orbital analysis at the dawn of 
broadly available \gaia\ data.
\end{abstract}

\keywords{\vspace{5em}}



\section{Introduction} 
\label{sec:introduction}

The Milky Way is considered to be our best laboratory for understanding galaxy
formation and evolution.  This premise hinges on the ability to precisely measure 
the astrometry and chemistry for (many) individual stars, and to use those data 
to infer the structure, kinematics, and chemical enrichment of the Galaxy 
\citep[e.g.,][]{Schlaufman_2009,Deason_2011,Ness_2012,Ness_2013a,Ness_2013b,
Casey_2012,Casey_2013,Casey_2014a,Casey_2014b,Kordopatis_2015,Bovy_2016}.  
However, these quantities are not known for even 1\% of stars in the Milky Way.  
Stellar distances are famously imprecise \citep[e.g.,][]{van_Leeuwen_2007,
Jofre_2015,Madler_2016}, proper motions can be plagued by unquantified systematics 
from the first epoch observations \citep[e.g.,][]{Casey_Schlaufman_2015}, and 
stellar spectroscopists frequently report significantly different chemical 
abundance patterns from the same spectrum \citep{Smiljanic_2014}.  The impact 
these issues have on scientific inferences cannot be understated.  Imperfect 
astrometry or chemistry limits understanding in a number of sub-fields in
astrophysics, including the properties of exoplanet host stars, the formation 
(and destruction) of stars and clusters, as well as the structure of the 
Galactic disk, to name a few.


These formed just some of the scientific drivers for the \gaia\ space telescope.
\gaia\ is primarily an astrometric mission, and will provide precise positions,
parallaxes and proper motions for more than $10^9$ stars in its final data
release in 2022.  While this is a sample size about four orders of magnitude 
larger than its predecessor \hipparcos, both astrometry and chemistry are 
required to fully characterize the formation and evolution of our Galaxy. 
\gaia\ will also provide radial velocities, stellar parameters and chemical 
abundances for a subset of brighter stars, but these measurements will not be 
available in the first few data releases. Until those abundances are available,
astronomers seeking to simultaneously use chemical and dynamical information are
reliant on ground-based spectroscopic surveys to complement the available 
\gaia\ astrometry.


The first \gaia\ data release will include the Tycho-Gaia Astrometric Solution
\citep[hereafter \tgas;][]{Michalik_2015a,Michalik_2015b}: positions, proper 
motions, and parallaxes for approximately two million stars in the Tycho-2 
\citep{Hog_2000} catalog.  After cross-matching all major stellar spectroscopic 
surveys\footnote{Specifically we cross-matched the Tycho-2 catalog against the 
\apogee\ \citep{Zasowski_2013}, \ges\ \citep{Gilmore_2012,Randich_2013}, 
\galah\ \citep{DeSilva_2015}, \lamost\ \citep{Cui_2012}, and \rave\ 
\citep{Steinmetz_2006} surveys}, we found the RAdial Velocity Experiment 
(\rave) survey to have the largest overlap with the first \gaia\ data release: 
292,036 stars.  We then used the \gaia\ universe model snapshot 
\citep{Robin_2012} to estimate the precision in parallax and proper motions that
will be available in the first \gaia\ data release (DR1) for stars in those 
overlap samples.  Comparing the expected precision to what is currently available, 
we further found that the \rave\ survey will benefit most from \gaia\ DR1.  The 
distances of 63\% of the \rave--\gaia\ DR1 overlap sample (182,862 stars) are 
expected to improve with the first \gaia\ data release, and 47\% of stars are 
likely to have better proper motions (137,211 stars).  Although the \gaia\ universe
model assumes end-of-mission uncertainties --- and does not account for systematics
in the first data release --- this calculation still provides intuition for the 
relative improvement the first \gaia\ data release can make to ground-based surveys.  
The expected improvements for \rave\ motivated us to examine what chemical abundances
were available from those data, and to evaluate whether we could enable new 
chemo-dynamic studies by contributing to the existing set of chemical abundances.


We briefly describe the \rave\ data in Section~\ref{sec:data}, before explaining
our methods in Section~\ref{sec:method}.  In Section~\ref{sec:validation}
we outline a number of validation experiments, including: internal sanity checks,
comparisons with literature samples, and investigations to ensure our results
are consistent with expectations from astrophysics.  We discuss the implications
of these comparisons and our results in Section~\ref{sec:discussion}.  We provide
concluding remarks in Section~\ref{sec:conclusion}, and describe how to access our
results electronically.


\section{Data}
\label{sec:data}


\rave\ is a magnitude-limited stellar spectroscopic survey of the (nearby) Milky Way,
principally designed to measure radial velocities for up to $10^6$ stars.
Observations were conducted on the 1.2~m UK Schmidt telescope at the Australian 
Astronomical Observatory\footnote{Formerly the Anglo-Australian Observatory} from 
2003--2013.  A large 5.7~degree field-of-view and robotic fibre positioner made for 
very efficient observing:  spectra for up to 150 targets could be simultaneously
acquired.  When observations concluded in April 2013, a total of \Nspectra\ useful 
spectra had been collected of \Nstars\ unique objects. 


The target selection for \rave\ is based on the $I$-band apparent magnitude,
$9 < I < 12$, with a weak $J - K_s > 0.5$ cut near the disk and bulge \citep{Wonjo_2016}.  
The $I$ band was used for the target selection because it has a good overlap with the
wavelength range that \rave\ operates in:  8410--8795~\AA.  This spectral region 
is very well-studied.  It is one of the key setups used for the ground-based 
high-resolution \ges\ survey \citep{Gilmore_2012,Randich_2013}, and the 
\rave\ resolution and wavelength coverage is comparable to the Radial Velocity 
Spectrometer on board the \gaia\ space telescope \citep{Kordopatis_2011,Recio-Blanco_2016}.  
The region includes the \ion{Ca}{2} near infrared triplet lines --- strong 
transitions that are dominated by pressure broadening --- which are visible even 
in metal-poor stars or spectra with very low signal-to-noise (S/N) ratios.
Atomic transitions of 
light-, $\alpha$-, and Fe-peak elements are also present, allowing for detailed 
chemical abundance studies.


The exposure times for \rave\ observations were optimised to obtain radial 
velocities for as many stars as possible.  Detailed chemical abundances were
always an important science goal of the survey, but this was a secondary objective.  
For this reason the distribution of S/N ratios in \rave\ spectra is considerably 
lower than other stellar spectroscopic surveys where chemical abundances are the 
primary motivation.  The \rave\ spectra have an effective resolution 
$\mathcal{R} \approx 7{,}500$ and the distribution of S/N ratios peaks at 
$\approx$50~pixel$^{-1}$.  For comparison, the \galah\ survey 
\citep{DeSilva_2015} --- which was specifically constructed for detailed chemical 
abundance analyses --- includes a wavelength range about 2.5 times larger at 
resolution $\mathcal{R} \approx 28{,}000$, and yet the \galah\ project still 
targets for S/N $\gtrsim100$.


Despite the relatively low resolution and S/N of the spectra compared to other
surveys, the \rave\ data releases have demonstrated the team's enormous success 
in deriving radial velocities, stellar astrophysical parameters ($\teff$, $\logg$),
as well as individual chemical abundances \citep{Steinmetz_2006,Zwitter_2008,
Siebert_2011,Kordopatis_2013,Kunder_2016}.  In this work we make use of spectra
that has been reprocessed for the fifth \rave\ data release.  These re-processing
steps include: a detailed re-reduction of all the original data frames, with flux
variances propagated at every step; an updated continuum-normalization procedure;
as well as revised determinations of stellar radial velocities and morphological
classifications. At the end of this processing for each survey observation we were 
provided with: rest-frame wavelengths (without resampling), continuum-normalized 
fluxes, $1\sigma$ uncertainties in the continuum-normalized flux values, as well 
as relevant metadata for each observation.  We refer the reader to the official 
fifth data release paper of the \rave\ survey, as presented by \citet{Kunder_2016}, 
for more details of this re-processing.


Given the high-quality of the normalization performed by the \rave\ team, we chose
not to re-normalize the spectra.  Our tests demonstrated that the procedure 
outlined in \citet{Kunder_2016} is sufficient for our analysis procedure. Therefore,
there is a limited number of pre-processing steps that we performed before starting
our analysis.  First, we calculated inverse variance arrays from the $1\sigma$ 
uncertainties provided, and then we re-sampled the flux and inverse variance
arrays onto a common rest-wavelength map for all stars.  Depending on the fibre 
used and the stellar radial velocity, the range of rest-frame wavelength values
varied for each star.  Given that fluxes were unavailable in the edge pixels for 
most stars, we excluded pixels outside of the rest wavelength range 
$8423.2\,{\rm \AA} \le \lambda \le 8777.6\,{\rm \AA}$.  This corresponds to about
30~pixels excluded on either side of the common wavelength array, leaving us with
945~pixels per spectrum for science.  


\section{Method}
\label{sec:method}


We chose to adopt a data-driven model --- rather than a physics-based model ---
for this analysis.  Specifically, we will use an implementation of \thecannon\
\citep{Ness_2015,Ness_2016}.  Although this choice complicated the construction 
of our model (e.g., see Section \ref{sec:the-training-set}), a data-driven approach 
makes use of all available information in the spectrum and lowers the S/N ratio 
at which systematic effects begin to dominate.  In other words, in the low S/N
regime, a well-constructed data-driven model will yield more precise labels 
(e.g., stellar parameters and chemical abundances) than most physics-driven 
models\footnote{However, see \citet{Casey_2016a}}.  This is particularly relevant 
for the low-resolution \rave\ data analysed here, because about half of the 
spectra have S/N $\lesssim 50$.


There are two main analysis steps when using \thecannon: the \emph{training} 
step and the \emph{test} step.  We describe these stages in the context of our
model in the following section, and a more thorough introduction can be found
in \citet{Ness_2015}.  We make the following explicit assumptions about the 
\rave\ spectra and \thecannon:

\begin{itemize}
\item We assume that any fibre- and time-dependent variations in spectral
resolution in the \rave\ spectra are negligible.
\item The \rave\ noise variances are approximately correct, independent between
pixels, and normally distributed.
\item We assume that the normalization procedure employed by the \rave\ pipeline
is invariant with respect to the labels we seek to measure (e.g., $\teff$, $\logg$,
or [Fe/H]), and invariant with respect to the S/N ratio.
\item We assume that stars with similar labels ($\teff$, $\logg$, and abundances)
have similar spectra.
\item A stellar spectrum is a smooth function of the label values for that star,
and we assume that the function is smooth enough within a sub-space of the labels
(e.g., the giant branch or the main-sequence) that it can be reasonably approximated 
with a low-order polynomial in label space.
\item The training set (Section~\ref{sec:the-training-set}) has mean accurate labels.
Note that we do not assume that \emph{every} label in the training is accurate -- we
can afford to have a small fraction of inaccurate labels (e.g., a few obvious 
misclassifications in the training set are affordable).
\item We assume the training set is representative of the test set (the \rave\ survey).
\end{itemize}


\subsection{The model}
\label{sec:the-model}


\noindent{}Given our assumptions, the model we adopt is
\begin{eqnarray}\label{eq:model}
y_{jn} & = & \vecv(\ell_n)\cdot\vectheta_j + e_{jn}\quad ,
\end{eqnarray}

\noindent{}where $y_{jn}$ is the pseudo-continuum-normalized flux for star $n$ at wavelength pixel
$j$, $\vecv(\ell_n)$ is the vectorizing function that takes as input the $K$ labels
$\ell_n$ for star $n$ and outputs functions of those labels as a vector of length
$D>K$, $\vectheta_j$ is a vector of length $D$ of parameters influencing the model at
wavelength pixel $j$, and $e_{jn}$ is the residual (noise).  Here we will only consider
vectorizing functions with polynomial expansions (e.g., $\teff^2$, see Sections 
\ref{sec:a-simple-model}-\ref{sec:evolved-star-model}).  The noise values $e_{jn}$ can 
be considered to be drawn from a Gaussian distribution with zero mean and variance 
$\sigma_{jn}^2 + s_j^2$, where $\sigma_{jn}^2$ is the variance in flux $y_{jn}$ and 
$s_j^2$ describes the excess variance at the $j$-th wavelength pixel. 


At the \emph{training} step we fix the $K$-lists of labels for the $n$ training set.
At each wavelength pixel $j$, we then find the parameters $\vectheta_j$ and $s_j^2$
by optimizing the penalized likelihood function
\begin{eqnarray}\label{eq:train}
\vectheta_j,s^2_j &\leftarrow& \argmin{\vectheta,s}\left[
    \sum_{n=0}^{N-1} \frac{[y_{jn}-\vecv(\ell_n)\cdot\vectheta]^2}{\sigma^2_{jn}+s^2}
    + \sum_{n=0}^{N-1} \ln(\sigma^2_{jn}+s^2) + \Lambda{}Q(\vectheta)
    \right]
  \quad ,
\end{eqnarray}

\noindent{}where $\Lambda$ is a regularization parameter which we will heuristically set
in later sections, and $Q(\vectheta)$ is a regularizing function that encourages 
$\vectheta$ values to take on zero values without breaking convexity:
\begin{eqnarray}\label{eq:regularization-function}
	Q(\vectheta) = \sum_{d=1}^{D-1} |{\theta_d}| \quad .
\end{eqnarray}

Note that the $d$ subscript here is zero-indexed; the function $Q(\vectheta)$ does not act
on the (first) $\theta_0$ coefficient, as this is a `pivot point' (mean flux value) that 
we do not expect to diminish with increasing regularization (e.g., see equation 
\ref{eq:vectorizer-three-label}).  In practice we first fix $s_j^2 = 0$ to make equation
\ref{eq:train} a convex optimization problem, then we optimize for $\vectheta_j$, before 
solving for $s_j^2$.  


The \emph{test step} is where we fix the parameters $\vectheta_j,s_j^2$ at all wavelength
pixels $j$, and optimize the $K$-list of labels $\ell_m$ for each test set star $m$.  Here
the objective function is:
\begin{eqnarray}\label{eq:test}
  \ell_m &\leftarrow& \argmin{\ell}\left[
    \sum_{j=0}^{J-1} \frac{[y_{jm}-\vecv(\ell)\cdot\vectheta_j]^2}{\sigma_{jm}^2 + s_j^2}
    \right]
  \quad .
\end{eqnarray}

After optimizing equation \ref{eq:test} for the $m$-th star we store the covariance matrix 
$\bm{\Sigma}_m$ for the labels $\ell_m$, which provides us with the formal errors on $\ell_m$. 
The formal errors are expected to be underestimated, and in Section~\ref{sec:validation} 
we judge the veracity of these errors through validation experiments.



\subsection{The training set}
\label{sec:the-training-set}


We sought to construct a training set of stars across the main-sequence, the
sub-giant branch, and the red giant branch.  We demanded stars with precisely measured
effective temperature $\teff$, surface gravity $\logg$, and elemental abundances
of O, Mg, Si, Ca, Al, Fe, and Ni.  This proved to be difficult because the magnitude
range of \rave\ does not overlap substantially with high-resolution spectroscopic
surveys.  The fourth internal data release of the \ges\ survey includes 
giant and main-sequence stars, but only 142 overlap with \rave, which is too small to
be a useful training set for our purposes.  The thirteenth data release from the 
\project{Sloan Digital Sky Survey} \citep{sloan_dr13} includes labels for \apogee\ stars on the
giant branch and (uncalibrated values for) the main-sequence, but our tests indicated
that the \apogee\ main-sequence labels suffered from significant systematic effects.  
A flat, then `up-turning' main-sequence is present, and the metallicity gradient trends in 
the opposite direction with respect to $\logg$ on the main-sequence (i.e., metal-poor
stars incorrectly sit above an isochrone in a classical Hertzsprung-Russell diagram).
If we consider lower-resolution studies as potential training sets, there are 2,369
stars that overlap with \lamost\ --- of which 2,213 have positive S/N ratios in the 
$g$-band (\texttt{snrg}).  However, the labels are expectedly less precise given the
lower resolution, there are no elemental abundances available for the main-sequence 
stars\footnote{Abundance information is available for \lamost\ stars from \citet{Ho_2016},
but that sample contains only giant stars}, and the \lamost\ lower main-sequence suffers
from the same systematic effects seen in \apogee\ data.  The \galah\ survey has a good 
overlap with \rave, but results from that sample are not yet precise enough for our 
purposes.


These constraints forced us to construct a heterogeneous training set.  Given previous
successes in transferring high S/N ratio labels from \apogee\ \citep{Ness_2015,
Ness_2016,Ho_2016,Casey_2016b}, we chose to use the 1,355 stars in the \apogee---\rave\ 
overlap sample for giant star labels in the training set.  Of these, about 900 are 
giants according to \apogee.  From this sample we selected stars to have: metallicity
determinations ([Fe/H] $> -5$); S/N ratios of $>$200 in \apogee\ and $>$25 in \rave; and 
we further required that \aspcap\ did not report any peculiar flags 
(\texttt{ASPCAPFLAG = 0}).  These restrictions left us with 536 stars along the giant 
branch, with metallicities ranging from $[{\rm Fe/H}] = -1.79$ to 0.26.  Intermediate 
tests with globular cluster members showed that the metallicity range of the training 
set needed to extend below $[{\rm Fe/H}] \lesssim -2$ in order for our catalog to be 
practically useful.  \stub{Supplemented this sample with known metal-poor stars in \rave}.



Assembling a suitable training set for the main-sequence and sub-giant branch was less
trivial.  There are no spectroscopic studies that extend the range of stellar types we 
are interested in (e.g., FGKM-type stars), and also have a large enough sample size that
overlaps with \rave.  Moreover, most of the spectroscopic studies we considered also 
showed a flat lower main-sequence, a systematic consequence of the analysis method adopted 
\citep[see][for discussion on this issue]{Bensby_2014}.  For these reasons we chose to make 
use of the \epic\ catalog \citep{Huber_2016} for the training set labels on the 
main-sequence and sub-giant branch.  The \epic\ catalog follows from the successful
\project{Kepler} input catalog \citep{Brown_2011}, and provides probabilistic stellar 
classifications for 138,600 stars in the \project{K2} fields based on the 
astrometric, asteroseismic, photometric, and spectroscopic information available for
every star.  There are 4,611 stars that overlap between \epic\ and \rave.


\epic\ differs from the \project{Kepler} input catalog because \epic\ does not 
benefit from having narrow-band $DDO_{51}$ photometry in order to aid dwarf/giant 
classification.  Despite this limitation the labels in the \epic\ catalog have 
already been shown to be accurate and trustworthy \citep{Huber_2016}.  However, 
when the posteriors are wide (i.e., the quoted confidence intervals are large) 
due to limited information available, it is possible that a star has been 
misclassified.  This is most prevalent for sub-giants, where \citet{Huber_2016} 
note that $\approx55-70$\% of sub-giants are misclassified as dwarfs.  The 
probability of misclassification is usually quantified in the uncertainties given
for each star; most dwarfs that have a higher likelihood of being sub-giants have
large confidence intervals.  Therefore, requiring low uncertainties will decrease 
the total sample size, but in practice it removes most misclassifications.  The 
situation is far more favourable for dwarfs and giants.  Only $1-4$\% of giant 
stars are misclassified as dwarfs, and about 7\% of dwarfs are misclassified as 
giants.  To summarise, the \epic\ labels with narrow confidence intervals are 
usually of high fidelity, and given that we have spectra, we can identify any
spurious misclassifications.


We sought to have a small overlap between our giant and main-sequence star training
sets.  Most of our giant training set is encapsulated within $0 < \logg < 3.5$, 
however there is a sparse sampling of stars reaching to $\logg \approx 4$.  We
required $\logg > 3.5$ for the \epic\ main-sequence/sub-giant star training set,
allowing for $\approx0.5$~dex of overlap between the two training sets.  We further
employed the following quality constraints: the upper and lower confidence intervals 
in $\teff$ must be below 150~K; the upper and lower confidence intervals in $\logg$ 
must be less than 0.15~dex; the $S/N$ of the \rave\ spectra must exceed 
30~pixel$^{-1}$; and $\teff \leqslant 6750$~K.  Unfortunately these strict constraints
removed most metal-poor stars, which we later found to cause the test labels to have
under-predicted abundances for dwarfs of low metallicity.  For this reason we relaxed
(ignored) those quality constraints for stars with $[{\rm Fe/H}] < -1$.  
\stub{Include known metal-poor stars in \rave?}. After training
a model based on main-sequence and giant stars (Section \ref{sec:the-model}), we found 
we could identify misclassifications by leave-one-out cross-validation.  However, we 
chose not to do this because the number of likely misclassifications in the training
set was negligible ($\approx1$\%), and the improvement in main-sequence test set labels
was minimal.  The distilled sample of the \rave--\epic\ overlap catalog contains 583 
stars (of 4,611).


\subsection{A 3-label model ($\teff$, $\logg$, $\feh$) for all stars}
\label{sec:a-simple-model}


We have constructed a justified training set for stars across the main-sequence, sub-giant,
and red giant branch.  However the lack of overlap between \rave\ and other works have
resulted in a somewhat peculiar situation.  Detailed abundances are available from \apogee\
for all giant stars in our sample, but only imprecise metallicities are available from
\epic\ for stars on the main-sequence and the sub-giant branch.  Here we will construct 
a simple model for \emph{all} stars that only makes use of three labels ($\teff$, $\logg$, 
[Fe/H]), before we outline how we derive abundances for giant branch stars.  The complexity
for this model will be quadratic ($\teff^2$ is the highest term), where the vectorizer 
$\vecv(\ell_n)$ expands as,
\begin{eqnarray}\label{eq:vectorizer-three-label}
\vecv(\ell_n) \rightarrow \left[1, T_{{\rm eff},n}, \logg_n, [{\rm Fe/H}]_n, T_{{\rm eff},n}^2, \logg_n\cdot{}T_{{\rm eff},n}, \feh_n\cdot{}T_{{\rm eff},n}, \logg_n^2, \feh_n\cdot\logg_n, \feh_n^2\right]
\end{eqnarray}

\noindent{}such that $\vecv(\ell)$ produces the design matrix:
\begin{eqnarray}
	\vecv(\ell) \rightarrow \begin{bmatrix} \vecv(\ell_0) \\ \vdots \\ \vecv(\ell_{N-1}) \end{bmatrix} \quad .
\end{eqnarray}


We used no regularization ($\Lambda = 0$) for this model.  After training the model we
treated all \Nspectra\ spectra as test set objects.  In the right-hand panel of Figure 
\ref{fig:tes-set-density} we show the effective temperature $\teff$ and surface gravity 
$\logg$ for all spectra.  The main-sequence and red giant branch are clearly visible.  
However, the details of stellar evolution are no longer present: the sub-giant branch is 
not discernible, and there are a number of systematic artefacts (over-densities) present
in label space.  These artefacts disappear when we require additional quality constraints 
(e.g., no peculiar morphological classifications), but the complexity of the 
Hertzsprung-Russell diagram is still not present.  Thus, we concluded that while this 
model could be useful for deriving stellar classifications (e.g., F2-type giant), the 
labels are too imprecise.


We chose to adopt separate models for the main-sequence and the red giant branch rather
than switch to a single model with more complexity.  This choice allowed us to derive
stellar parameters for stars on the main-sequence and sub-giant branch, as well as 
detailed elemental abundances for red giant branch stars.  However, adopting two separate
models introduces the challenge of how to combine the results from two models, or how to
assign one star as `belonging' to a single model.  In Section~\ref{sec:joining-the-models}
we describe how we will use the 3-label model of the main-sequence and giant branch
(introduced in this section) to discriminate between results from a 3-label 
main-sequence model in Section \ref{sec:unevolved-star-model} and a 9-label giant 
star model in Section \ref{sec:evolved-star-model}.


\subsection{A 3-label model ($\teff$, $\logg$, $\feh$) for unevolved stars}
\label{sec:unevolved-star-model}


We constructed a three-label quadratic model using only main-sequence and sub-giant
stars in the \epic\ training set.  In order to set the regularization hyperparameter
$\Lambda$ for this model, we trained 30 models with different regularization strengths,
spaced evenly in logarithmic steps between $\Lambda = 10^{-3}$ to $\Lambda = 10^{3}$.
We then performed leave-one-out cross-validation for each model.  Specifically, for 
each star in the training set: we removed the star; trained the model; and then 
inferred labels from the removed star as if it was a test object. We also performed 
leave-one-out cross-validation on an unregularized ($\Lambda = 0$) model, which we 
will use as the basis for comparison.  For every model we calculated the bias and
root-mean-square (RMS) deviation between the training set labels and those inferred
through cross-validation. 


We show the \emph{percentage difference} in the RMS deviation of the labels with respect
to the unregularized model in Figure \ref{fig:set-hyperparameters}.  The upper 
and lower envelope represent the boundaries across all labels, showing that with increasing
regularization, the RMS decreased in \emph{all} labels.  We found similar
improvements in the biases, however these were already minimal in the unregularized
case.  The improvement in RMS reaches a minimum value near $\Lambda = X.XX$, where
%3.56224789e+01
we achieve RMS deviations that are about 30\% better than the unregularized case.
Based on this improvement we set $\Lambda = X.XX$ for this model.  At this regularization 
strength, the bias and RMS values found by leave-one-out
cross validation are, respectively: $X$~K and $XX$~K for $\teff$, $X.XX$~dex and $X.XX$~dex
for $\logg$, with $X.XX$~dex and $X.XX$~dex for [Fe/H].


We inferred labels for all \rave\ spectra using this main-sequence/sub-giant star model.  
The results for the survey sample are shown in the left panel of Figure \ref{fig:test-set-density}.  
The increased density of Solar-type stars is consistent with \rave\
observing stars in the local neighbourhood, and the high number of turn-off
and main-sequence stars relative to the sub-giant branch is expected from 
the relative lifetimes of these evolutionary phases.  An over-density
of stars near the giant branch clump is also present.  This artefact is due 
to having giant stars in the test set, but not in the training set, and the
model is (poorly) extrapolating outside the convex hull of the training set.


\subsection{A 9-label model for detailed abundances of giant stars}
\label{sec:evolved-star-model}

The red giant branch stars in our training set have have stellar parameters 
($\teff$, $\logg$) and up to 15 elemental abundances from the \aspcap\ 
\citep{Garcia_Perez_2016}.  A subset of these elements have atomic transitions in the 
\rave\ wavelength region: \ion{O}{1}, \ion{Mg}{1}, \ion{Al}{1}, \ion{Si}{1}, 
\ion{Ca}{2}, \ion{Ti}{1}, \ion{Fe}{1}, and \ion{Ni}{1}.  However, we 
excluded [Ti/H] from our abundance list because of systematics in the \aspcap\
[Ti/H] abundances \citep{Holtzman_2015,Hawkins_2016}.  Therefore we are left 
with nine labels in our giant star model: $\teff$, $\logg$, and seven elemental 
abundances.  


Similar to Sections \ref{sec:a-simple-model} and \ref{sec:unevolved-star-model},
we used a quadratic vectorizer for the giant star model.  Here the terms are 
expanded in the same way as equation \ref{eq:vectorizer-three-label}, only
with nine labels instead of three.  We set the regularization hyperparameter $\Lambda$
in the same way described in Section \ref{sec:unevolved-star-model}, using the same 30 trials of $\Lambda$.
The results are shown in Figure \ref{fig:set-hyperparameters}, where
again the enveloped region represents the minimum and maximum change in RMS label
deviation with respect to the unregularized case.  At the point of maximum improvement 
near $\Lambda = X.XX$, the RMS in all nine labels has decreased by at least $X$\%, 
and up to $X$\%.  Here the regularization strength also produces a sparser matrix
of $\vectheta$, with $\approx20$\% more terms (mostly cross-terms) having zero-valued entries.
Based on the increased model sparsity and decreasing RMS deviation in the labels, 
we adopt $\Lambda = X.XX$ for the giant star model.


Results from cross-validation on our regularized model imply that the precision
in labels is XXX~K in $\teff$, X.XX~dex in $\logg$, and between X.XX~dex (EL) and 
X.XX~dex (EL) depending on the elemental abundance.  We inferred labels for all
\Nspectra\ \rave\ spectra using this model, and these results are summarized in
Figure \ref{fig:test-set-density} (middle panel).  The red clump is clearly visible
and in the expected location, without requiring any post-analysis calibration.
Artefacts due to dwarf stars in the test set are also present.


\subsection{Joining the models}
\label{sec:joining-the-models}

We have derived labels for all \Nspectra\ \rave\ spectra using the three models
described in previous sections.  The results from the joint model --- that includes 
the main-sequence, sub-giant and red giant branch --- shows that a single quadratic
model is too simple for the \rave\ spectral range.  The other models have problems,
too: unrealistic over-densities in label space show that the main-sequence model and
the giant model make very poor extrapolations for stars outside their respective
training sets.  For these reasons we were forced to exclude or severely penalize 
incorrect results from both models.


Before attempting to join the results from different models, we excluded results
in either model that had a reduced $\chi_{r}^2 > 3$.  We further discarded stars with
labels that are outside the extent of the training set.  Specifically for the
results from the giant model we (conservatively) excluded stars with derived 
$\logg > 3.5$, and for the results from the main-sequence model we excluded 
sub-giant stars ($\logg < 4$ and $\teff < 5000$~K) that were outside the 
two-dimensional ($\teff$, $\logg$) convex hull of the training set used for the main-sequence model.  Unfortunately these restrictions did not remove all spurious
results.  The reason for this can be explained with an example:  consider that our 
giant star model was trained with only giant stars but tested with both giant 
stars and dwarf stars.  Some classes of stars (e.g., metal-poor dwarfs) can 
project into a region of label space that would suggest it is a giant 
(e.g., a clump star).  These objects could have relatively low $\chi_{r}^2$ values
(e.g., $\chi_{r}^2 < 3$) and in this example, they would appear as bonafide red clump 
stars.  These incorrect projections are extrapolation errors in high dimensions that
project to `normal' parts of the label space in two dimensions.  For these reasons 
we also made use of the joint model in Section~\ref{sec:a-simple-model}
to inform whether we should adopt results from: the red giant branch model; the
main-sequence/sub-giant model; or a linear combination of the two. 

In Figure \ref{fig:joint-model-differences} we show the differences in effective temperature
$\teff$ and surface gravity $\logg$ between: the main-sequence model and the joint
model; and the differences between the red giant branch model and the joint model. We have scaled
the differences in $\teff$ and $\logg$ to make the central peak near $(0, 0)$ to 
% ARC to HOGG: This is a total hack. Should we fit for these and \rho_{\teff,\logg} or are you content with the hack?
be approximately isotropic by setting
	$\delta_{T_{\rm eff}} = 90$~K for the main-sequence model, 
	$\delta_{T_{\rm eff}} = 50$~K for the giant model, and 
	$\delta_{\log{g}} = 0.15$~dex for both models.
The stars within the peak at $(0, 0)$ represent objects where
the joint model and the comparison model both report similar labels.  The artefacts 
seen in the Hertszprung-Russell diagrams in Figure \ref{fig:test-set-density} are
also present in Figure \ref{fig:joint-model-differences} as over-densities far away
from the central peak.  Therefore, we can 
adopt the scaled distance in labels $\teff$ and $\logg$ from the joint model,
% ARC to HOGG: Please suggest any nomenclature or definition changes as you see fit!
\begin{eqnarray}
	d_{ms} & = & \left(\frac{T_{{\rm eff},ms} - T_{{\rm eff},joint}}{\delta_{T_{{\rm eff},ms}}}\right)^2 + \left(\frac{\log{g}_{ms} - \log{g}_{joint}}{\delta_{\log{g},ms}}\right)^2   \nonumber \\
	d_{giant} & = & \left(\frac{T_{{\rm eff},giant} - T_{{\rm eff},joint}}{\delta_{T_{{\rm eff},giant}}}\right)^2 + \left(\frac{\log{g}_{giant} - \log{g}_{joint}}{\delta_{\log{g},giant}}\right)^2   \quad ,
\end{eqnarray}

\noindent{}as a metric to derive the weights,
\begin{eqnarray}
	w_{ms} = \frac{1}{{d_{ms}}^2} & \text{and} & w_{giant} = \frac{1}{{d_{giant}}^2} \quad ,
\end{eqnarray}

\noindent{}and produce the weighted labels $\hat\ell$:
\begin{eqnarray}
	\hat\ell = \frac{w_{ms}\ell_{ms} + w_{giant}\ell_{giant}}{w_{ms} + w_{giant}} \quad .
\end{eqnarray}

We calculate weighted errors of $\hat\ell$ and correlation coefficients in 
the same manner.  In Figure \ref{fig:model-weights} we show the mean relative
weight $w_{ms}/(w_{ms} + w_{giant})$ within each two-dimensional bin of 
$\hat\teff$ and $\hat\logg$.  For giant stars the relative weight of 
the main-sequence model is zero, and vice-versa for main-sequence stars.
The relative weights smoothly transition from 0 to 1 on the sub-giant branch
near $\log{g} \approx 3.5$, in the training set overlap region of both models.
For abundance labels in the giant model that are not in the main-sequence
model (e.g., [O/H], [Mg/H]), we only report abundances if there was no contribution
from the main-sequence model: $w_{ms} = 0$.


Our catalog now contains entries of $XXX$ stars with effective temperature $\teff$,
surface gravity $\logg$, iron abundance [Fe/H], and $XXX$ of those stars are giants
with elemental abundances of [O/H], [Al/H], [Mg/H], [Ca/H], [Si/H], and [Ni/H].
In total, this leads to a catalog of X.XX million stellar abundances. 
The weighted $\teff$ and $\logg$ values for stars meeting different
S/N constraints are shown in Figure \ref{fig:test-set-hrd}, both in logarithmic density
and mean metallicity.  The artefacts from individual models are no longer 
apparent, and the rich structure of the Hertzsprung-Russell is visible.

\subsection{Spectrophotometric distances}
\label{sec:distances}

\stub{These will be derived by Jason Sanders when he returns from his honeymoon.
The timing means that the distances will not be available when we submit the first
version to arXiv. So we may need to remove this section and the reference to it
from the abstract.}

\section{Validation experiments}
\label{sec:validation}


In addition to the cross-validation tests that we have previously described, 
we have conducted a number of internal and external validation experiments to 
test the validity of our results.  We will begin by describing internal validation
tests based on repeat observations, before evaluating our accuracy based on
high-resolution literature comparisons.


\subsection{Multi-epoch observations}
\label{sec:repeat-observations}

The \rave\ survey performed repeat observations for 43,918 stars with time 
intervals ranging from a few hours to up to four years.  This timing was 
constructed to be quasi-logarithmic such that spectroscopic binaries could
be optimally identified. Most of the stars that were observed multiple times
were only observed twice, with thirteen visits being the maximum number 
of observations for any target.  These repeat observations allow us to 
quantify the level of (in)correctness in our formal errors.  For every star
with multiple visits we constructed a high S/N stacked spectrum for that
star, weighted by the inverse variances in the individual visit spectra.


We treated the stacked spectra as normal survey stars.  We inferred labels
using all models and joined the results as per Section \ref{sec:joining-the-models}.  The labels inferred from the stacked spectra served as the basis
of comparison for all individual visits to that object.  Figure 
\ref{fig:formal-errors-comparison} shows the difference in labels 
between the comparison spectrum and a repeat visit, normalized by their 
formal errors summed in quadrature (e.g., 
$\Delta\logg/\sqrt{\sigma_{\logg,stacked}^2 + \sigma_{\logg,visit}^2}$).
If our measurements were unbiased by S/N and the formal errors were 
representative, these values should be normally distributed with a zero 
mean and unit variance.
\stub{Are they? (Probably). How much should we inflate our formal errors, and should this be as a function of S/N? Note that in all Figures and tables,
we show these inflated errors only, as we gauge them to be more representative.}


\subsection{External validation}
\label{sec:external-validation}

\subsubsection{Comparison with \rave\ DR5}

Here we cross-match our results against literature samples to verify that
our results are accurate.  Because this work is an independent analysis, our
first point of reference is against the official \rave\ data release.  This
comparison is shown in Figure \ref{fig:rave-dr5-comparison}, 
where we show our results with respect to the fifth \rave\ data release \citep{Kunder_2016}.  In order
to provide a fair comparison, we only show the $XXX$ stars that meet a number of quality
flags in both samples: 
\stub{What selections qualify as a fair comparison in both samples?}
There is very good agreement in $\teff$, with a bias
and RMS of just XX~K and XXX~K, respectively.  We find our $\logg$ values are
closest to the `calibrated' \rave\ surface gravities, which make use of 
asteroseismic information to correct (increase) the $\logg$ values along the
giant branch.  The calibrated \rave\ surface gravities also correct the 
location of the red clump (by $\sim0.5$~dex) to be in the same position in our
results.  However, while our $\logg$ values generally agree with the calibrated
\rave\ values along the giant branch, our $\logg$ values do differ slightly 
along the main-sequence.  This is principally because the main-sequence in 
this work tends to taper down towards higher $\logg$ values at cooler 
temperatures, whereas the comparison sample tend to have a slightly flatter
main-sequence.  This difference is not likely to have a very significant 
effect on the detailed abundance or spectrophotometric distance determinations
between these studies \citep{Binney_2014}.

\subsubsection{Comparison with the \citet{Kordopatis_2013} calibration sample}
\label{sec:validation-kordopatis}

\subsubsection{Comparisons with Reddy, Bensby, and Valenti \& Fischer}
\label{sec:validation-gold-standards}

\stub{Caution re MP dwarfs appear as sub-giants in our sample}

\subsubsection{Comparison with the \ges\ survey}
\label{sec:validation-ges}

There are 142 stars that overlap between \rave\ and the fourth internal
data release of the \ges\ survey. These are a mix of main-sequence, 
sub-giant and red giant branch stars.  Despite most of these stars having
relatively low S/N ratios in \rave\ ($\approx 25$), there is good 
agreement in stellar parameters (Figure~\ref{fig:ges-comparison}.  
The RMS in effective temperature, surface gravity and metallicity is
236~K, 0.39~dex, and 0.18~dex, respectively. 
% TODO: Check if those numbers above are still valid when final catalog is used.

\stub{A paragraph about the abundance agreement.}



\subsection{Astrophysical validation}
\label{sec:astrophysical-validation}


After verifying that our stellar labels are comparable with other
high-resolution studies, here we verify that the abundances we find
are consistent with expectations from astrophysics.  One example is stars in
globular clusters, which show anti-correlations in light element abundances 
\citep[e.g.,][and references therein]{Norris_Da_Costa_1995,Carretta_2009}.  In the \rave\
survey, \cite{Anguiano_2015} found 70 stars with positions and radial velocities that are
consistent with being members of globular clusters: 49 stars belonging to NGC~5139 ($\omega$~Centauri), 11 members of the retrograde
globular cluster NGC~3201, and 10 members of NGC~362.  
All of these systems are well-studied and exhibit abundance anti-correlations 
in Na--O and Mg--Al \citep{Marino_2011,Carretta_2009,Carretta_2013,Munoz_2013}.  Other correlations are also present 
(e.g., C--N), but those relationships are less relevant for this work because 
we do not measure C and N abundances here.  We note that \emph{none} of these 
globular cluster stars are present in our training set.



We show the detailed abundances for these globular cluster members in
Figure \ref{fig:globular-cluster-abundances}.  For comparison purposes we
have included abundances from the high-resolution studies of \citep{people}.
\stub{What do we find?}

% Assuming we recover these relationships...
Indeed, this demonstrates that our abundance labels can be used to identify
globular cluster members that are now tidally disrupted \citep{Anguiano_2016,Kuzma_2016,Navin_2016},
even for stars with low S/N ratios.

% Some literature notes:
% See http://iopscience.iop.org/article/10.1088/0004-637X/731/1/64/pdf
% NGC 3201 has Na-O anti-correlation: Carretta et al. (2009)
% NGC 362 has "two discrete groups along the Na-O anti-correlation": https://arxiv.org/abs/1307.4085
% NGC 362 has Mg-Al relationship (low Mg scatter, large Al scatter) --> Carretta (2013)
% NGC 362 has a large metallicity spread in RAVE (Anguiano_2015) but not the case in high-res studies (e.g. https://arxiv.org/abs/1307.4085 has a metallicity spread of 0.05/0.08 dex) may indicate better cluster selection needed and can be done with cannon results.
% NGC 3201 has Mg-Al relationship (http://arxiv.org/pdf/1305.3645.pdf and references therein)



% Astrophysical validation:
% --> Residuals as a function of galactic position? (DIBS)



\section{Discussion}
\label{sec:discussion}


\stub{Chemistry + Kinematics = Stockholm}

\stub{What do the two provide us with?}

\stub{What is one interesting science result should we show based on these abundances and Hipparcos results, which was not known previously?}


\section{Conclusion}
\label{sec:conclusion}

We performed an independent analysis of \rave\ spectra using a data-driven approach.
We report effective temperature $\teff$, surface gravity $\logg$, and metallicity [Fe/H]
for X stars on the main-sequence and sub-giant branch.  For another X stars on the red
giant branch, we deliver $\teff$, $\logg$, and a total of X.XX~million elemental abundances:
O, Mg, Al, Ca, Si, Fe, and Ni for X stars.  

\stub{Typical precision}

\stub{A key science result that we see already}

\stub{Something about exoplanet host stars, TESS, etc}

This catalog constitutes the largest collection of stellar abundances for stars in the
first \gaia\ data release.  When combined with positions and 3D velocities from \gaia,
we hope the \project{unRAVE} catalog will help advance understanding of the Milky Way's
formation and evolution.


\subsection*{Access the results electronically}

% TODO: Replace ZENODO-URL with a data url
\noindent\colorbox{unoffensive-warning}{\parbox{\dimexpr\linewidth-2\fboxsep}{
\noindent{}Source code for this project is available at \texttt{\giturl}\hspace{-0.5em},
and this document was compiled from revision hash \texttt{\githash} in that repository.
Inferred labels, covariance matrices, and relevant metadata are available electronically
at \texttt{\url{ZENODO-URL}} \citep{DATA_REPOSITORY}.  Please note that it is a condition
of using these results that the fifth \rave\ data release by \citet{Kunder_2016} must 
also be cited, as the work presented here would not have been possible without the 
tireless efforts of the entire \rave\ collaboration, past and present.
}}




\acknowledgements
We thank 
	Jonathan Bird (Vanderbilt),
	Sven Buder (MPIA), 
and 
	Sergey Koposov (Cambridge).
This research made use of: 
  	NASA's Astrophysics Data System Bibliographic Services;
  	Astropy, a community-developed core Python package for Astronomy \citep{astropy};
and 
  	\project{TOPCAT} \citep{Taylor_2005}.
This work was partly supported by the European Union FP7 programme through ERC 
grant number 320360. Funding for K.~H.~ has been provided through the Simons 
Foundation Society of Fellows and the Marshall Scholarship.
Funding for RAVE has been provided by: the Australian Astronomical Observatory; 
the Leibniz-Institut fuer Astrophysik Potsdam (AIP); the Australian National 
University; the Australian Research Council; the French National Research Agency;
the German Research Foundation (SPP 1177 and SFB 881); the European Research 
Council (ERC-StG 240271 Galactica); the Istituto Nazionale di Astrofisica at 
Padova; The Johns Hopkins University; the National Science Foundation of the USA
(AST-0908326); the W. M. Keck foundation; the Macquarie University; the 
Netherlands Research School for Astronomy; the Natural Sciences and Engineering 
Research Council of Canada; the Slovenian Research Agency; the Swiss National 
Science Foundation; the Science \& Technology Facilities Council of the UK; 
Opticon; Strasbourg Observatory; and the Universities of Groningen, Heidelberg 
and Sydney. The RAVE web site is https://www.rave-survey.org.  

\begin{thebibliography}{dummy}
\bibitem[Anguiano et al.(2015)]{Anguiano_2015} Anguiano, B., Zucker, D.~B., Scholz, R.-D., et al.\ 2015, \mnras, 451, 1229 

\bibitem[Anguiano et al.(2016)]{Anguiano_2016} Anguiano, B., De Silva, G.~M., Freeman, K., et al.\ 2016, \mnras, 457, 2078 

\bibitem[Astropy Collaboration et 
al.(2013)]{astropy} Astropy Collaboration, Robitaille, T.~P., Tollerud, E.~J., et al.\ 2013, Astronomy \& Astrophysics, 558, AA33

\bibitem[Binney et al.(2014)]{Binney_2014} Binney, J., Burnett, B., Kordopatis, G., et al.\ 2014, \mnras, 437, 351 

\bibitem[Bensby et al.(2014)]{Bensby_2014} Bensby, T., Feltzing, S., \& Oey, M.~S.\ 2014, \aap, 562, A71 

\bibitem[Bovy et al.(2016)]{Bovy_2016} Bovy, J., Rix, H.-W., Schlafly, E.~F., et al.\ 2016, \apj, 823, 30 

\bibitem[Brown et al.(2011)]{Brown_2011} Brown, T.~M., Latham, D.~W., Everett, M.~E., \& Esquerdo, G.~A.\ 2011, \aj, 142, 112 

\bibitem[Carretta et al.(2009)]{Carretta_2009} Carretta, E., Bragaglia, A., Gratton, R.~G., et al.\ 2009, \aap, 505, 117 

\bibitem[Carretta et al.(2013)]{Carretta_2013} Carretta, E., Bragaglia, A., Gratton, R.~G., et al.\ 2013, \aap, 557, A138 

\bibitem[Casey et al.(2012)]{Casey_2012} Casey, A.~R., Keller, S.~C., \& Da Costa, G.\ 2012, \aj, 143, 88 

\bibitem[Casey et al.(2013)]{Casey_2013} Casey, A.~R., Da Costa, G., Keller, S.~C., \& Maunder, E.\ 2013, \apj, 764, 39 

\bibitem[Casey et al.(2014a)]{Casey_2014a} Casey, A.~R., Keller, S.~C., Da Costa, G., Frebel, A., \& Maunder, E.\ 2014, \apj, 784, 19 

\bibitem[Casey et al.(2014b)]{Casey_2014b} Casey, A.~R., Keller, S.~C., Alves-Brito, A., et al.\ 2014, \mnras, 443, 828 

\bibitem[Casey \& Schlaufman(2015)]{Casey_Schlaufman_2015} Casey, A.~R., \& Schlaufman, K.~C.\ 2015, \apj, 809, 110 

\bibitem[Casey(2016)]{Casey_2016a} Casey, A.~R.\ 2016, \apjs, 223, 8 

\bibitem[Casey et al.(2016)]{Casey_2016b} Casey, A.~R., Hogg, D.~W., Ness, M., et al.\ 2016, arXiv:1603.03040 

\bibitem[Cui et al.(2012)]{Cui_2012} Cui, X.-Q., Zhao, Y.-H., Chu, Y.-Q., et al.\ 2012, Research in Astronomy and Astrophysics, 12, 1197 

\bibitem[De Silva et al.(2015)]{DeSilva_2015} De Silva, G.~M., Freeman, K.~C., Bland-Hawthorn, J., et al.\ 2015, \mnras, 449, 2604 

\bibitem[Deason et al.(2011)]{Deason_2011} Deason, A.~J., Belokurov, V., \& Evans, N.~W.\ 2011, \mnras, 416, 2903 

\bibitem[Garc{\'{\i}}a P{\'e}rez et al.(2016)]{Garcia_Perez_2016} Garc{\'{\i}}a P{\'e}rez, A.~E., Allende Prieto, C., Holtzman, J.~A., et al.\ 2016, \aj, 151, 144 

\bibitem[Gilmore et al.(2012)]{Gilmore_2012} Gilmore, G., Randich, S., Asplund, M., et al.\ 2012, The Messenger, 147, 25

\bibitem[Hawkins et al.(2016)]{Hawkins_2016} Hawkins, K., Masseron, T., Jofre, P., et al.\ 2016, arXiv:1604.08800 

\bibitem[Holtzman et al.(2015)]{Holtzman_2015} Holtzman, J.~A., Shetrone, M., Johnson, J.~A., et al.\ 2015, \aj, 150, 148 

\bibitem[Ho et al.(2016)]{Ho_2016} Ho, A.~Y.~Q., Ness, M.~K., Hogg, D.~W., et al.\ 2016, arXiv:1602.00303 
 
\bibitem[H{\o}g et al.(2000)]{Hog_2000} H{\o}g, E., Fabricius, C., Makarov, V.~V., et al.\ 2000, \aap, 355, L27 

\bibitem[Huber et al.(2016)]{Huber_2016} Huber, D., Bryson, S.~T., Haas, M.~R., et al.\ 2016, \apjs, 224, 2 

\bibitem[Jofr{\'e} et al.(2015)]{Jofre_2015} Jofr{\'e}, P., M{\"a}dler, T., Gilmore, G., et al.\ 2015, \mnras, 453, 1428 

\bibitem[Kordopatis et al.(2011)]{Kordopatis_2011} Kordopatis, G., Recio-Blanco, A., de Laverny, P., et al.\ 2011, \aap, 535, A106 

\bibitem[Kordopatis et al.(2013)]{Kordopatis_2013} Kordopatis, G., Gilmore, G., Steinmetz, M., et al.\ 2013, \aj, 146, 134 

\bibitem[Kordopatis et al.(2015)]{Kordopatis_2015} Kordopatis, G., Binney, J., Gilmore, G., et al.\ 2015, \mnras, 447, 3526 

\bibitem[Kunder et al.(2016)]{Kunder_2016} Kunder, A., et al.\ 2016, submitted

\bibitem[Kuzma et al.(2016)]{Kuzma_2016} Kuzma, P.~B., Da Costa, G.~S., Mackey, A.~D., \& Roderick, T.~A.\ 2016, \mnras, 461, 3639 

\bibitem[M{\"a}dler et al.(2016)]{Madler_2016} M{\"a}dler, T., Jofr{\'e}, P., Gilmore, G., et al.\ 2016, arXiv:1606.03015 

\bibitem[Marino et al.(2011)]{Marino_2011} Marino, A.~F., Milone, A.~P., Piotto, G., et al.\ 2011, \apj, 731, 64 

\bibitem[Michalik et al.(2015a)]{Michalik_2015a} Michalik, D., Lindegren, L., \& Hobbs, D.\ 2015, \aap, 574, A115 

\bibitem[Michalik et al.(2015b)]{Michalik_2015b} Michalik, D., Lindegren, L., Hobbs, D., \& Butkevich, A.~G.\ 2015, \aap, 583, A68 

\bibitem[Mu{\~n}oz et al.(2013)]{Munoz_2013} Mu{\~n}oz, C., Geisler, D., \& Villanova, S.\ 2013, \mnras, 433, 2006 

\bibitem[Navin et al.(2016)]{Navin_2016} Navin, C.~A., Martell, S.~L., \& Zucker, D.~B.\ 2016, arXiv:1606.06430 

\bibitem[Ness et al.(2012)]{Ness_2012} Ness, M., Freeman, K., Athanassoula, E., et al.\ 2012, \apj, 756, 22 

\bibitem[Ness et al.(2013a)]{Ness_2013a} Ness, M., Freeman, K., Athanassoula, E., et al.\ 2013, \mnras, 430, 836 

\bibitem[Ness et al.(2013b)]{Ness_2013b} Ness, M., Freeman, K., Athanassoula, E., et al.\ 2013, \mnras, 432, 2092 

\bibitem[Ness et al.(2015)]{Ness_2015} Ness, M., Hogg, D.~W., Rix, H.-W., Ho, A.~Y.~Q., \& Zasowski, G.\ 2015, \apj, 808, 16 

\bibitem[Ness et al.(2016)]{Ness_2016} Ness, M., Hogg, D.~W., Rix, H.-W., et al.\ 2016, \apj, 823, 114 

\bibitem[Norris \& Da Costa(1995)]{Norris_Da_Costa_1995} Norris, J.~E., \& Da Costa, G.~S.\ 1995, \apj, 447, 680 

\bibitem[Randich et al.(2013)]{Randich_2013} Randich, S., Gilmore, G., \& Gaia-ESO Consortium 2013, The Messenger, 154, 47 

\bibitem[Recio-Blanco et al.(2016)]{Recio-Blanco_2016} Recio-Blanco, A., de Laverny, P., Allende Prieto, C., et al.\ 2016, \aap, 585, A93 

\bibitem[Reddy et al.(2003)]{Reddy_2003} Reddy, B.~E., Tomkin, J., Lambert, D.~L., \& Allende Prieto, C.\ 2003, \mnras, 340, 304 

\bibitem[Reddy et al.(2006)]{Reddy_2006} Reddy, B.~E., Lambert, D.~L., \& Allende Prieto, C.\ 2006, \mnras, 367, 1329 

\bibitem[Robin et al.(2012)]{Robin_2012} Robin, A.~C., Luri, X., Reyl{\'e}, C., et al.\ 2012, \aap, 543, A100 

\bibitem[Schlaufman et al.(2009)]{Schlaufman_2009} Schlaufman, K.~C., Rockosi, C.~M., Allende Prieto, C., et al.\ 2009, \apj, 703, 2177 

\bibitem[Siebert et al.(2011)]{Siebert_2011} Siebert, A., Williams, M.~E.~K., Siviero, A., et al.\ 2011, \aj, 141, 187 

\bibitem[SDSS Collaboration et al.(2016)]{sloan_dr13} SDSS Collaboration, Albareti, F.~D., Allende Prieto, C., et al.\ 2016, arXiv:1608.02013 

\bibitem[Smiljanic et al.(2014)]{Smiljanic_2014} Smiljanic, R., Korn, A.~J., Bergemann, M., et al.\ 2014, \aap, 570, A122 

\bibitem[Steinmetz et al.(2006)]{Steinmetz_2006} Steinmetz, M., Zwitter, T., Siebert, A., et al.\ 2006, \aj, 132, 1645 

\bibitem[Taylor(2005)]{Taylor_2005} Taylor, M.~B.\ 2005, Astronomical Data Analysis Software and Systems XIV, 347, 29 

\bibitem[Valenti \& Fischer(2005)]{Valenti_Fischer_2005} Valenti, J.~A., \& Fischer, D.~A.\ 2005, \apjs, 159, 141 

\bibitem[van Leeuwen(2007)]{van_Leeuwen_2007} van Leeuwen, F.\ 2007, \aap, 474, 653 

\bibitem[Wonjo et al.(2016)]{Wonjo_2016} Wonjo, J., et al.\ 2016, submitted

\bibitem[Zasowski et al.(2013)]{Zasowski_2013} Zasowski, G., Johnson, J.~A., Frinchaboy, P.~M., et al.\ 2013, \aj, 146, 81 

\bibitem[Zwitter et al.(2008)]{Zwitter_2008} Zwitter, T., Siebert, A., Munari, U., et al.\ 2008, \aj, 136, 421 

\end{thebibliography}

\clearpage

\begin{figure}[p]
\includegraphics[width=\textwidth]{figures/hrd-train-set.pdf}
\caption{A H-R diagram showing the training set labels.\label{fig:training-set-hrd}}
\end{figure}



\begin{figure*}[p]
\includegraphics[width=\textwidth]{figures/test-set-density.pdf}
\caption{\label{fig:test-set-density}}
\end{figure*}


\begin{figure}[p]
\includegraphics[width=\textwidth]{figures/set-giant-regularization.pdf}
\caption{The fractional improvement in RMS deviation between inferred and training labels for the 9-label giant model at different regularization strengths.  The RMS values were calculated by leave-one-out cross-validation.  The points and solid line indicate the mean improvement across all labels. The filled area represents the minimum and maximum improvements across all labels.  The maximum improvement (decrease) in RMS label deviations occurs at $\Lambda \approx 0.3$, and this improvement persists until $\Lambda \approx 0.6$.\label{fig:set-hyperparameters}}
\end{figure}



\begin{figure}[p]
\center
\includegraphics[width=0.5\textwidth]{figures/joint-model-differences.pdf}
\caption{The normalized differences in effective temperature $\teff$ and surface gravity $\logg$ between the main-sequence model and the joint model (top panel), and the giant model and the joint model (bottom panel).  The density scaling is logarithmic, and the differences in $\teff$ and $\logg$ are scaled to make them approximately isotropic (see text for details).  The peak at $(0, 0)$ represents good agreement between the joint model and comparison model, whereas the over-densities elsewhere are a consequence of testing the model on stars very different to the training set (e.g., dwarf stars tested on a model trained with only giant stars).\label{fig:joint-model-differences}}
\end{figure}


\begin{figure}[p]
\center
\includegraphics[width=0.5\textwidth]{figures/model-weights.pdf}
\caption{The mean relative main-sequence model weight $w_{ms}/(w_{ms} + w_{giant})$ at each hexagonal bin of weighted effective temperature $\teff$ and surface gravity $\logg$.  The relative weighting illustrates how only results from the main-sequence model are adopted for unevolved stars, and there is a gradual transition to using results from the giant model, before only results from the giant model are used for evolved stars.\label{fig:model-weights}}
\end{figure}




\begin{figure}[p]
\includegraphics[width=\textwidth]{figures/hrd-test-set.pdf}
\caption{The effective temperature $\teff$ and surface gravity $\logg$ for \rave\ stars after combining labels from the main-sequence and giant star models.  Only stars with $v\sin{i} < 1$~km~s$^{-1}$ and $\chi_{red} < 3$ are shown.  The top three panels show logarithmic density, and bins in the lower three panels are colored by the median metallicity in each bin.\label{fig:test-set-hrd}}
\end{figure}






\begin{figure*}[p]
\caption{The RMS of the test set labels as a function of S/N ratio for repeated stars in the test set.\label{fig:test-set-repeats}}
\end{figure*}

\begin{figure*}[p]
\includegraphics[width=\textwidth]{figures/dr4-comparison.png}
\caption{Stellar label comparison between the \project{UNRAVE} catalog and the \project{RAVE} fourth data release \citep{Kordopatis_2013}.\label{fig:rave-dr4-comparison}}
\end{figure*}



\begin{figure*}[p]
\includegraphics[width=\textwidth]{figures/ges-comparison.pdf}
\caption{Stellar label comparison between the \project{unRAVE} catalog and the \ges\ fourth internal data release.\label{fig:ges-stellar-parameters}}
\end{figure*}


\begin{figure*}[p]
\includegraphics[width=\textwidth]{figures/ges-abundances.pdf}
\caption{Individual chemical abundances in \project{unRAVE} compared to \ges.\label{fig:ges-abundances}}
\end{figure*}


\begin{figure*}[p]
\includegraphics[width=\textwidth]{figures/gold-standard-comparison.pdf}
\caption{Stellar parameter comparisons for stars in common between this work and studies using high-resolution, high S/N spectra and \hipparcos\ parallaxes.\label{fig:gold-standard-comparison}}
\end{figure*}

\begin{figure*}[p]
\includegraphics[width=\textwidth]{figures/gold-standard-hrd.pdf}
\caption{HRD of gold standard comparisons.\label{fig:gold-standard-hrd}}
\end{figure*}


\begin{figure*}[p]
\includegraphics[width=\textwidth]{figures/kordopatis-calibration.pdf}
\caption{Comparison with the calibration (literature) samples used by \citet{Kordopatis_2013} and \citet{Kunder_2016}.\label{fig:kordopatis-calibration}}
\end{figure*}

\begin{figure}[p]
\includegraphics[width=0.5\textwidth]{figures/gce.pdf}
\caption{Giant star abundances ([X/Fe]) with respect to [Fe/H], showing the Galactic chemical evolution trend for individual elements.\label{fig:gce}}
\end{figure}




\end{document}
