%% This file is part of the UNRAVE Project 
%% Copyright 2016 the authors.  All rights reserved.

\documentclass[preprint2,trackchanges]{aastex}

\usepackage{amsmath,amssymb}
\usepackage{bm}

\newcommand{\project}[1]{\textsl{#1}}
\newcommand{\thecannon}{\project{The~Cannon}}
\newcommand{\acronym}[1]{{\small{#1}}}
\newcommand{\rave}{\project{\acronym{RAVE}}}
\newcommand{\logg}{\log g}
\newcommand{\teff}{T_{\mathrm{eff}}}
\newcommand{\argmin}[1]{\underset{#1}{\operatorname{argmin}}\,}

%\AuthorCallLimit=10

\begin{document}

\title{The \project{UNRAVE} catalog}

\author{Andrew R. Casey}
\affil{Institute of Astronomy, Madingley Road, Cambridge CB3 0HA}

\author{Some combination of: Gerry Gilmore, Keith Hawkins, David W. Hogg, Gal Matijevic, Melissa Ness, Hans Walter-Rix, and the \project{RAVE} collaboration}
%\altaffiltext{3}{\texttt{arc@ast.cam.ac.uk}}

\begin{abstract}
On 14 September 2016 the \project{Gaia} mission will release positions, proper motions, and parallaxes for about two million stars.  Most studies intending to make use of those data would benefit enormously from stellar chemical abundances.
  The RAdial Velocity Experiment (\project{RAVE}) survey has acquired spectra for approximately 280,000 stars in common with the first \project{Gaia} data release, constituting the largest overlap of any stellar spectroscopic sample.  Here we perform an independent analysis of the \project{RAVE} spectra using a new non-parametric version of \thecannon.  We deliver stellar astrophysical parameters ($\teff$, $\logg$) and chemical abundances of up to 8 elements (Fe, Na, Mg, Al, C, N, Ti, Ca) for X stars.  The typical precision in abundances is $X.XX$~dex.  We validate our results internally from repeat visits, and externally against existing high-resolution spectroscopic surveys. 
\end{abstract}

\keywords{}

\section{Introduction} 
\label{sec:introduction}


\section{Methods}
\label{sec:methods}


\section{Validation}
\label{sec:validation}


\section{Conclusion}
\label{sec:conclusion}


\acknowledgements
% We thank,......

This research made use of: Astropy, a community-developed core Python package for
Astronomy \citep{astropy}, NASA's Astrophysics Data System Bibliographic Services;
 and \project{TOPCAT} \citep{Taylor2005}.
This work was partly supported by the European Union FP7 programme through ERC 
grant number 320360.
Funding for RAVE has been provided by: the Australian Astronomical Observatory; the Leibniz-Institut fuer Astrophysik Potsdam (AIP); the Australian National University; the Australian Research Council; the French National Research Agency; the German Research Foundation (SPP 1177 and SFB 881); the European Research Council (ERC-StG 240271 Galactica); the Istituto Nazionale di Astrofisica at Padova; The Johns Hopkins University; the National Science Foundation of the USA (AST-0908326); the W. M. Keck foundation; the Macquarie University; the Netherlands Research School for Astronomy; the Natural Sciences and Engineering Research Council of Canada; the Slovenian Research Agency; the Swiss National Science Foundation; the Science \& Technology Facilities Council of the UK; Opticon; Strasbourg Observatory; and the Universities of Groningen, Heidelberg and Sydney. The RAVE web site is at https://www.rave-survey.org.  



\begin{thebibliography}{dummy}
\bibitem[Astropy Collaboration et 
al.(2013)]{astropy} Astropy Collaboration, Robitaille, T.~P., Tollerud, E.~J., et al.\ 2013, Astronomy \& Astrophysics, 558, AA33

\bibitem[Taylor(2005)]{Taylor2005} Taylor, M.~B.\ 2005, Astronomical Data Analysis Software and Systems XIV, 347, 29 

\end{thebibliography}

\clearpage

%\begin{figure}[p]
%\caption{Choose two wavelengths (one continuum and one interesting)
%  and plot some scatter plots of flux vs various parameters, and also
%  cross-validation results for the
%  hyper-parameters.\label{fig:onewavelength}}
%\end{figure}


\end{document}
