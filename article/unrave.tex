%% This file is part of the UNRAVE Project 
%% Copyright 2016 the authors.  All rights reserved.


%% TODO:
%% Co-authors: Look for \stub{}s. If you can expand it, do so!

\documentclass[preprint,trackchanges]{aastex}

\usepackage{amsmath}
\usepackage{bm}

%\usepackage{xcolor}
%\definecolor{unoffensive-warning}{HTML}{B4DCED}

% This section can be removed at submission.
% ------------------------------------------
\usepackage{color}
\usepackage{datenumber}

\newcounter{dateone}
\newcounter{datetwo}

\newcommand{\difftoday}[3]{%
  \setmydatenumber{dateone}{\the\year}{\the\month}{\the\day}%
  \setmydatenumber{datetwo}{#1}{#2}{#3}%
  \addtocounter{datetwo}{-\thedateone}%
  \the\numexpr(\thedatetwo)\relax\space days}
% ------------------------------------------

\IfFileExists{vc.tex}{\input{vc.tex}}{\newcommand{\githash}{UNKNOWN}\newcommand{\giturl}{UNKNOWN}}

\newcommand{\acronym}[1]{{\small{#1}}}

\newcommand{\project}[1]{\textsl{#1}}
\newcommand{\gaia}{\project{Gaia}}
\newcommand{\thecannon}{\project{The~Cannon}}
\newcommand{\rave}{\project{\acronym{RAVE}}}
\newcommand{\galah}{\project{\acronym{GALAH}}}
\newcommand{\ges}{\project{Gaia-ESO}}
\newcommand{\apogee}{\project{\acronym{APOGEE}}}
\newcommand{\aspcap}{\project{\acronym{ASPCAP}}}
\newcommand{\lamost}{\project{\acronym{LAMOST}}}
\newcommand{\hipparcos}{\project{Hipparcos}}
\newcommand{\epic}{\project{K2/EPIC}}
\newcommand{\sdss}{\project{\acronym{SDSS}}}
\newcommand{\tgas}{\project{\acronym{TGAS}}}
\newcommand{\unrave}{\project{unRAVE}}

\newcommand{\stub}[1]{{\color{blue} \textbf{#1}}}

\newcommand{\teff}{T_{\mathrm{eff}}}
\newcommand{\logg}{\log g}
\newcommand{\feh}{[\mathrm{Fe/H}]}

\newcommand{\Nspectra}{520,782}
\newcommand{\Nstars}{457,589}
\newcommand{\Nstarsqc}{447,207} % v0.95 finite values
\newcommand{\Nabundances}{1,692,825} %v0.95 finite values


\newcommand{\Dvector}[1]{\boldsymbol{#1}}
\newcommand{\vectheta}{\Dvector{\theta}}
\newcommand{\vecv}{\Dvector{v}}
\newcommand{\argmin}[1]{\underset{#1}{\operatorname{argmin}}\,}


% For AASTeX v6
%\AuthorCallLimit=10
%\fullcollaborationName{The \rave\ Collaboration}

\begin{document}
% Remove at submission:
\slugcomment{{\color{red} \textbf{To appear on arXiv on 13 Sep 2016 (submission in \difftoday{2016}{9}{9})}}}


\title{The \project{unRAVE} catalog}

\author{Andrew R. Casey}
\affil{Institute of Astronomy, Madingley Road, Cambridge CB3 0HA}

% Gilmore, Matijevic, McMillan, Siviero, Wojno

% Specific contributors to this work:
% There is some order already in mind, but final order will be determined by  contribution.
\author{Some combination of: Harry Enke, Kenneth C. Freeman, Gerry Gilmore, Keith Hawkins, David W. Hogg, Georges Kordopatis, Gal Matijevic, Melissa Ness, Jason Sanders, Matthias Steinmetz, Hans Walter-Rix}

% RAVE DR5 core team:
\author{Luca Casagrande, Andrea Kunder, Paul McMillan, Alessandro Siviero, Jennifer Wojno, Toma\v{z} Zwitter}
% Zwitter:  University of Ljubljana, Faculty of Mathematics and Physics, Jadranska 19, 1000 Ljubljana, Slovenia

% RAVE builders
\author{Joss Bland-Hawthorn, Brad K. Gibson, Arnaud Siebert, Olivier Bienayme, Julio F. Navarro, Ulisse Munari, Warren Reid, Rosemary Wyse, Amina Helmi}
% Bland-Hawthorn: Sydney Institute for Astronomy, School of Physics, University of Sydney, NSW 2006, Australia.
% Gibson: E.A. Milne Centre for Astrophysics, University of Hull, Hull, HU6 7RX, United Kingdom.
% Siebert: Observatoire astronomique de Strasbourg, Universit\'e de Strasbourg, CNRS, UMR 7550, 11 rue de l?Universit\'e, F-67000 Strasbourg, France
% Bienayme: Observatoire astronomique de Strasbourg, Universit\'e de Strasbourg, CNRS, UMR 7550, 11 rue de l?Universit\'e, F-67000 Strasbourg, France
% Navarro: Senior CIfAR Fellow; Department of Physics and Astronomy, University of Victoria, Victoria, BC, Canada V8P 5C2
% Munari: INAF Astronomical Observatory of Padova, 36012 Asiago (VI), Italy
% Reid (1/2): Department of Physics and Astronomy, Macquarie University, Sydney, NSW 2109, Australia
% Reid (1/2): Western Sydney University, Penrith South DC, NSW 1797
% Wyse: Johns Hopkins University, Baltimore, MD, USA
% Helmi: Kapteyn Astronomical Institute, University of Groningen, P.O. Box 800, 9700 AV Groningen, The Netherlands
\author{and the \rave\ collaboration}

\begin{abstract}
The Milky Way is a powerful laboratory for understanding galaxy formation,
as it can provide orbits, ages, atmospheric parameters, as well as chemical 
abundances for vast sets of individual stars.  These inferences require 
both astrometric and spectroscopic data.  Indeed, in order to be maximally
useful for chemo-dynamic studies, the Tycho-Gaia Astrometric Solution 
(\tgas) sample in the first \gaia\ data release requires a spectroscopic 
complement that includes radial velocities, stellar parameters, and 
elemental abundances.  Among existing spectroscopic samples, the RAdial 
Velocity Experiment (\rave) survey has the largest overlap with \tgas: 
up to 292,036 stars.  Here we present a data-driven re-analysis of \rave\ 
spectra, using an implementation of \thecannon, that yields more precise 
and accurate stellar parameters and abundances than previous \rave\
data releases.  We constructed our model using high-fidelity \apogee\ 
stellar parameters and abundances for red giant branch stars that overlap
with \rave, and stellar parameters from the \project{K2}/\project{Ecliptic 
Pole Input Catalog} for main-sequence and sub-giant branch stars.
We derive, and validate, improved effective temperature $\teff$, surface 
gravity $\logg$, and chemical abundances of up to seven elements (O, Mg, 
Al, Si, Ca, Fe, Ni) for \Nstarsqc\ stars.  The typical precision in 
% TODO
chemical abundances is X.XX~dex for red giant branch stars.  The \unrave\ 
catalog presented here, in conjunction with \tgas, represents the 
most powerful data set for chemo-orbital analysis at the dawn of broadly 
available \gaia\ data.
\end{abstract}

\keywords{}



\section{Introduction} 
\label{sec:introduction}

The Milky Way is considered to be our best laboratory for understanding galaxy
formation and evolution.  This premise hinges on the ability to precisely measure 
the astrometry and chemistry for (many) individual stars, and to use those data 
to infer the structure, kinematics, and chemical enrichment of the Galaxy 
\citep[e.g.,][]{Nordstrom_2004,Schlaufman_2009,Deason_2011,Casagrande_2011,Ness_2012,Ness_2013a,Ness_2013b,
Casey_2012,Casey_2013,Casey_2014a,Casey_2014b,Boeche_2013,Kordopatis_2015,Bovy_2016}.  
However, these quantities are not known for even 1\% of stars in the Milky Way.  
Stellar distances are famously imprecise \citep[e.g.,][]{van_Leeuwen_2007,
Jofre_2015,Madler_2016}, proper motions can be plagued by unquantified systematics 
from the first epoch observations \citep[e.g.,][]{Casey_Schlaufman_2015}, and 
stellar spectroscopists frequently report significantly different chemical 
abundance patterns from the same spectrum \citep{Smiljanic_2014}.  The impact 
these issues have on scientific inferences cannot be understated.  Imperfect 
astrometry or chemistry limits understanding in a number of sub-fields in
astrophysics, including the properties of exoplanet host stars, the formation 
(and destruction) of stars and clusters, as well as the study of stellar
populations and Galactic structure, to name a few.


The \gaia\ mission represents a critical step forward in understanding the Galaxy.
\gaia\ is primarily an astrometric mission, and will provide precise positions,
parallaxes and proper motions for more than $10^9$ stars in its final data
release in 2022.  While this is a sample size about four orders of magnitude 
larger than its predecessor \hipparcos, both astrometry and chemistry are 
required to fully characterize the formation and evolution of our Galaxy. 
\gaia\ will also provide radial velocities, stellar parameters and chemical 
abundances for a subset of brighter stars, but these measurements will not be 
available in the first few data releases. Until those abundances are available,
astronomers seeking to simultaneously use chemical and dynamical information are
reliant on ground-based spectroscopic surveys to complement the available 
\gaia\ astrometry.


The first \gaia\ data release will include the Tycho-Gaia Astrometric Solution
\citep[hereafter \tgas;][]{Michalik_2015a,Michalik_2015b}: positions, proper 
motions, and parallaxes for approximately two million stars in the Tycho-2 
\citep{Hog_2000} catalog.  After cross-matching all major stellar spectroscopic 
surveys\footnote{Specifically we cross-matched the Tycho-2 catalog against the 
\apogee\ \citep{Zasowski_2013}, \ges\ \citep{Gilmore_2012,Randich_2013}, 
\galah\ \citep{DeSilva_2015}, \lamost\ \citep{Cui_2012}, and \rave\ 
\citep{Steinmetz_2006} surveys.}, we found that the RAdial Velocity Experiment 
(\rave) survey is expected to have the largest overlap with the first \gaia\ 
data release: up to 292,036 stars.  We used the \gaia\ universe model snapshot 
\citep{Robin_2012} to estimate the precision in parallax and proper motions that
could be available in the first \gaia\ data release (DR1) for stars in those 
overlap samples.  Comparing the expected precision to what is currently available, 
we further found that the \rave\ survey will benefit most from \gaia\ DR1: 63\%
of stars in the \rave--\gaia\ DR1 overlap sample ($\approx$182,862 stars) are 
expected to improve with the first \gaia\ data release, and 47\% of stars ($\approx$137,211 stars) are 
likely to have better proper motions.  Although the \gaia\ universe
model assumes end-of-mission uncertainties --- and does not account for systematics
in the first data release --- this calculation still provides intuition for the 
relative improvement the first \gaia\ data release can make to ground-based surveys.  
The expected improvements for \rave\ motivated us to examine what chemical abundances
were available from those data, and to evaluate whether we could enable new 
chemo-dynamic studies by contributing to the existing set of chemical abundances.


We briefly describe the \rave\ data in Section~\ref{sec:data}, before explaining
our methods in Section~\ref{sec:method}.  In Section~\ref{sec:validation}
we outline a number of validation experiments, including: internal sanity checks,
comparisons with literature samples, and investigations to ensure our results
are consistent with expectations from astrophysics.  We discuss the implications
of these comparisons in Section~\ref{sec:discussion}, and conclude with instructions
for how to access our results electronically.


\section{Data}
\label{sec:data}


\rave\ is a magnitude-limited stellar spectroscopic survey of the (nearby) Milky Way,
principally designed to measure radial velocities for up to $10^6$ stars.
Observations were conducted on the 1.2~m UK Schmidt telescope at the Australian 
Astronomical Observatory\footnote{Formerly the Anglo-Australian Observatory.} from 
2003--2013.  A large 5.7~degree field-of-view and robotic fibre positioner made for 
very efficient observing:  spectra for up to 150 targets could be simultaneously
acquired.  When observations concluded in April 2013, at least \Nspectra\ useful 
spectra had been collected of more than \Nstars\ unique objects. 


The target selection for \rave\ is based on the $I$-band apparent magnitude,
$9 < I < 12$, with a weak $J - K_s > 0.5$ cut near the disk and bulge \citep{Wojno_2016}.  
The $I$ band was used for the target selection because it has a good overlap with the
wavelength range that \rave\ operates in:  8410--8795~\AA.  The resolution and 
wavelength coverage is comparable to the Radial Velocity Spectrometer on board 
the \gaia\ space telescope \citep{Munari_2005,Kordopatis_2011,Recio-Blanco_2016}, 
and the wavelength range overlaps with one of the key setups used for the ground-based 
high-resolution \ges\ survey \citep{Gilmore_2012,Randich_2013}.  The region 
includes the \ion{Ca}{2} near infrared triplet lines --- strong transitions that 
are dominated by pressure broadening --- which are visible even in metal-poor stars
or spectra with very low signal-to-noise (S/N) ratios.  Atomic transitions of 
light-, $\alpha$-, and Fe-peak elements are also present, allowing for detailed 
chemical abundance studies.


The exposure times for \rave\ observations were optimised to obtain radial 
velocities for as many stars as possible.  Detailed chemical abundances were
always an important science goal of the survey, but this was a secondary objective.  
For this reason the distribution of S/N ratios in \rave\ spectra is considerably 
lower than other stellar spectroscopic surveys where chemical abundances are the 
primary motivation.  The \rave\ spectra have an effective resolution 
$\mathcal{R} \approx 7{,}500$ and the distribution of S/N ratios peaks at 
$\approx$50~pixel$^{-1}$.  For comparison, the \galah\ survey 
\citep{DeSilva_2015} --- which was specifically constructed for detailed chemical 
abundance analyses --- includes a wavelength range about 2.5 times larger at 
resolution $\mathcal{R} \approx 28{,}000$, and yet the \galah\ project still 
targets for S/N $\gtrsim100$ per resolution element.


Despite the relatively low resolution and S/N of the spectra compared to other
surveys, the \rave\ data releases have provided excellent radial velocities, 
stellar astrophysical parameters ($\teff$, $\logg$), as well as individual 
chemical abundances \citep{Steinmetz_2006,Zwitter_2008,Siebert_2011,Boeche_2011,
Kordopatis_2013,Kunder_2016}.  In this work we make use of spectra
that has been reprocessed for the fifth \rave\ data release.  These re-processing
steps include: a detailed re-reduction of all the original data frames, with flux
variances propagated at every step; an updated continuum-normalization procedure;
as well as revised determinations of stellar radial velocities and morphological
classifications. At the end of this processing for each survey observation we were 
provided with: rest-frame wavelengths (without resampling), continuum-normalized 
fluxes, $1\sigma$ uncertainties in the continuum-normalized flux values, as well 
as relevant metadata for each observation.  We refer the reader to the official 
fifth data release paper of the \rave\ survey, as presented by \citet{Kunder_2016}, 
for more details of this re-processing.


Given the high-quality of the normalization performed by the \rave\ team, we chose
not to re-normalize the spectra.  Our tests demonstrated that the procedure 
outlined in \citet{Kunder_2016} is sufficient for our analysis procedure. Therefore,
there is a limited number of pre-processing steps that we performed before starting
our analysis.  First, we calculated inverse variance arrays from the $1\sigma$ 
uncertainties provided, and then we re-sampled the flux and inverse variance
arrays onto a common rest-wavelength map for all stars.  Depending on the fibre 
used and the stellar radial velocity, the range of rest-frame wavelength values
varied for each star.  Given that fluxes were unavailable in the edge pixels for 
most stars, we excluded pixels outside of the rest wavelength range 
$8423.2\,{\rm \AA} \le \lambda \le 8777.6\,{\rm \AA}$.  This corresponds to about
30~pixels excluded on either side of the common wavelength array, leaving us with
945~pixels per spectrum for science.  


\section{Method}
\label{sec:method}


We chose to adopt a data-driven model for this analysis, rather than the
physics-based models used in the \rave\ data releases.  Specifically, we will
use an implementation of \thecannon\
\citep{Ness_2015,Ness_2016}.  Although this choice complicated the construction 
of our model (e.g., see Section \ref{sec:the-training-set}), a data-driven approach 
makes use of all available information in the spectrum and lowers the S/N ratio 
at which systematic effects begin to dominate.  In other words, in the low S/N
regime, a well-constructed data-driven model will yield more precise \emph{labels}
(e.g., stellar parameters and chemical abundances) than most physics-driven 
models\footnote{However, see \citet{Casey_2016a}.}.  This is particularly relevant 
for the low-resolution \rave\ data analysed here, because about half of the 
spectra have S/N $\lesssim 50$~pixel$^{-1}$.


There are two main analysis steps when using \thecannon: the \emph{training} 
step and the \emph{test} step.  We describe these stages in the context of our
model in the following section, and a more thorough introduction can be found
in \citet{Ness_2015}.  We make the following explicit assumptions about the 
\rave\ spectra and \thecannon:

\begin{itemize}
\item We assume that any fibre- and time-dependent variations in spectral
resolution in the \rave\ spectra are negligible.
\item The \rave\ noise variances are approximately correct, independent between
pixels, and normally distributed.
\item We assume that the normalization procedure employed by the \rave\ pipeline
is invariant with respect to the labels we seek to measure (e.g., $\teff$, $\logg$,
or [Fe/H]), and invariant with respect to the S/N ratio.  In other words, we assume
that the normalization procedure does not produce different results for high S/N
spectra compared to low S/N spectra, nor does the normalization procedure vary 
non-linearly with respect to stellar parameters (e.g., [Fe/H]).
\item We assume that stars with similar labels ($\teff$, $\logg$, and abundances)
have similar spectra.
\item A stellar spectrum is a smooth function of the label values for that star,
and we assume that the function is smooth enough within a sub-space of the labels
(e.g., the giant branch or the main-sequence) that it can be reasonably approximated 
with a low-order polynomial in label space.
\item The training set (Section~\ref{sec:the-training-set}) has mean accurate labels
for most, but not all stars. That is to say that we do not assume that \emph{every} 
label in the training is accurate -- we
can afford to have a small fraction of inaccurate labels (e.g., a few obvious 
misclassifications in the training set are affordable).
\item We assume that the training data are similar (in spectra) to the test data 
where they overlap in label space, and we assume that the training data spans enough
of the label space to capture the variation in the test-set spectra.
\end{itemize}


\subsection{The model}
\label{sec:the-model}


\noindent{}Given our assumptions, the model we adopt is
\begin{eqnarray}\label{eq:model}
y_{jn} & = & \vecv(\ell_n)\cdot\vectheta_j + e_{jn}\quad ,
\end{eqnarray}

\noindent{}where $y_{jn}$ is the pseudo-continuum-normalized flux for star $n$ at wavelength pixel
$j$, $\vecv(\ell_n)$ is the vectorizing function that takes as input the $K$ labels
$\ell_n$ for star $n$ and outputs functions of those labels as a vector of length
$D>K$, $\vectheta_j$ is a vector of length $D$ of parameters influencing the model at
wavelength pixel $j$, and $e_{jn}$ is the residual (noise).  Here we will only consider
vectorizing functions with second-order polynomial expansions (e.g., $\teff^2$, see Sections 
\ref{sec:a-simple-model}-\ref{sec:evolved-star-model}).  The noise values $e_{jn}$ can 
be considered to be drawn from a Gaussian distribution with zero mean and variance 
$\sigma_{jn}^2 + s_j^2$, where $\sigma_{jn}^2$ is the variance in flux $y_{jn}$ and 
$s_j^2$ describes the excess variance at the $j$-th wavelength pixel. 


At the \emph{training} step we fix the $K$-lists of labels for the $n$ training set stars.
At each wavelength pixel $j$, we then find the parameters $\vectheta_j$ and $s_j^2$
by optimizing the penalized likelihood function
\begin{eqnarray}\label{eq:train}
\vectheta_j,s^2_j &\leftarrow& \argmin{\vectheta,s}\left[
    \sum_{n=0}^{N-1} \frac{[y_{jn}-\vecv(\ell_n)\cdot\vectheta]^2}{\sigma^2_{jn}+s^2}
    + \sum_{n=0}^{N-1} \ln(\sigma^2_{jn}+s^2) + \Lambda{}\,Q(\vectheta)
    \right]
  \quad ,
\end{eqnarray}

\noindent{}where $\Lambda$ is a regularization parameter which we will heuristically set
in later sections, and $Q(\vectheta)$ is a L1 regularizing function that encourages 
$\vectheta$ values to take on zero values without breaking convexity \citep{Casey_2016b}:
\begin{eqnarray}\label{eq:regularization-function}
	Q(\vectheta) = \sum_{d=1}^{D-1} |{\theta_d}| \quad .
\end{eqnarray}

Note that the $d$ subscript here is zero-indexed; the function $Q(\vectheta)$ does not act
on the (first) $\theta_0$ coefficient, as this is a `pivot point' (mean flux value) that 
we do not expect to diminish with increasing regularization (e.g., see equation 
\ref{eq:vectorizer-three-label}).  In practice we first fix $s_j^2 = 0$ to make equation
\ref{eq:train} a convex optimization problem, then we optimize for $\vectheta_j$, before 
solving for $s_j^2$.  


The \emph{test step} is where we fix the parameters $\vectheta_j,s_j^2$ at all wavelength
pixels $j$, and optimize the $K$-list of labels $\ell_m$ for the $m$-th test set star.  Here
the objective function is:
\begin{eqnarray}\label{eq:test}
  \ell_m &\leftarrow& \argmin{\ell}\left[
    \sum_{j=0}^{J-1} \frac{[y_{jm}-\vecv(\ell)\cdot\vectheta_j]^2}{\sigma_{jm}^2 + s_j^2}
    \right]
  \quad .
\end{eqnarray}

After optimizing equation \ref{eq:test} for the $m$-th star we store the covariance matrix 
$\bm{\Sigma}_m$ for the labels $\ell_m$, which provides us with the formal errors on $\ell_m$. 
The formal errors are expected to be underestimated, and in Section~\ref{sec:validation} 
we judge the veracity of these errors through validation experiments.



\subsection{The training set}
\label{sec:the-training-set}


We sought to construct a training set of stars across the main-sequence, the
sub-giant branch, and the red giant branch.  We required stars with precisely measured
effective temperature $\teff$, surface gravity $\logg$, and elemental abundances
of O, Mg, Si, Ca, Al, Fe, and Ni.  This proved to be difficult because the magnitude
range of \rave\ does not overlap substantially with high-resolution spectroscopic
surveys.  The fourth internal data release of the \ges\ survey includes 
giant and main-sequence stars, but only 142 overlap with \rave, which is too small to
be a useful training set for our purposes.  The thirteenth data release from the 
\project{Sloan Digital Sky Survey} \citep{sloan_dr13} includes labels for \apogee\ stars on the
giant branch and (uncalibrated values for) the main-sequence, but our tests indicated
that the \apogee\ main-sequence labels suffered from significant systematic effects.  
A flat, then `up-turning' main-sequence is present, and the metallicity gradient trends in 
the opposite direction with respect to $\logg$ on the main-sequence (i.e., metal-poor
stars incorrectly sit above an isochrone in a classical Hertzsprung-Russell diagram).
If we consider lower-resolution studies as potential training sets, there are 2,369
stars that overlap with \lamost\ --- of which 2,213 have positive S/N ratios in the 
$g$-band (\texttt{snrg}).  However, the labels are expectedly less precise given the
lower resolution, there are no elemental abundances available for the main-sequence 
stars\footnote{Abundance information is available for \lamost\ stars from \citet{Ho_2016},
but that sample contains only giant stars.}, and the \lamost\ lower main-sequence suffers
from the same systematic effects seen in \apogee\ data. 


These constraints forced us to construct a heterogeneous training set.  Given previous
successes in transferring high S/N ratio labels from \apogee\ \citep{Ness_2015,
Ness_2016,Ho_2016,Casey_2016b}, we chose to use the 1,355 stars in the \apogee---\rave\ 
overlap sample for giant star labels in the training set.  Of these, about 900 are 
giants according to \apogee.  From this sample we selected stars to have: 
determinations in all abundances of interest ([X/H] $> -5$ for O, Mg, Al, Si, Ca, Fe, and Ni); S/N ratios of $>$200 in \apogee\ and $>$25 in \rave; and 
we further required that \aspcap\ did not report any peculiar flags 
(\texttt{ASPCAPFLAG = 0}).  These restrictions left us with 536 stars along the giant 
branch, with metallicities ranging from $[{\rm Fe/H}] = -1.79$ to 0.26.  Intermediate 
tests with globular cluster members showed that the metallicity range of the training 
set needed to extend at least below $[{\rm Fe/H}] \lesssim -2$ in order for our catalog 
to be practically useful.  Without additional metal-poor stars, the lowest metallicity
labels reported by our model would be around $[{\rm Fe/H}] \approx -2$, even for well
studied stars with $[{\rm Fe/H}] \sim -4$ (e.g., CD~38-245).  For this reason we
supplemented our sample of \apogee\ giant stars with 176 known metal-poor giant stars 
observed by \rave.  The effective temperature $\teff$, surface gravity $\logg$ and
iron abundance [Fe/H] labels were adopted from \citet{Fulbright_2010, Ruchti_2011}.
The elemental abundances for O, Mg, Al, Si, Ca, and Ni were assumed to follow typical
trends of Galactic chemical evolution, such that we assumed $[{\rm Mg/Fe}] = +0.4$,
$[{\rm O/Fe}] = +0.4$, $[{\rm Al/Fe}] = -0.5$, $[{\rm Ca/Fe}] = +0.4$, 
$[{\rm Si/Fe}] = +0.4$, and $[{\rm Ni/Fe}] = -0.25$.  This decision is made solely 
to ensure that our overall metallicity scale extends that of the \rave\ survey, down 
to $[{\rm Fe/H}] \sim -4$.  In other words, we do not recommend the use of individual 
abundance labels at $[{\rm Fe/H}] \sim -4$; in practice the abundances of most of these
elements are largely unrecoverable in the \rave\ wavelength region at $[{\rm Fe/H}] \sim -4$.
We discuss this issue in more detail in Section~\ref{sec:discussion}.


Assembling a suitable training set for the main-sequence and sub-giant branch was less
trivial.  There are no spectroscopic studies that extend the range of stellar types we 
are interested in (e.g., FGKM-type stars), and which have a large enough sample size 
overlapping with \rave.  Moreover, most of the spectroscopic studies we considered also 
showed a flat lower main-sequence, a systematic consequence of the analysis method adopted 
\citep[see][for discussion on this issue]{Bensby_2014}.  For these reasons we chose to make 
use of the \epic\ catalog \citep{Huber_2016} for the training set labels on the 
main-sequence and sub-giant branch.  The \epic\ catalog follows from the successful
\project{Kepler} input catalog \citep{Brown_2011}, and provides probabilistic stellar 
classifications for 138,600 stars in the \project{K2} fields based on the 
astrometric, asteroseismic, photometric, and spectroscopic information available for
every star.  There are 4,611 stars that overlap between \epic\ and \rave.


\epic\ differs from the \project{Kepler} input catalog because \epic\ does not 
benefit from having narrow-band $DDO_{51}$ photometry in order to aid dwarf/giant 
classification.  Despite this limitation the labels in the \epic\ catalog have 
already been shown to be accurate and trustworthy \citep{Huber_2016}.  However, 
when the posteriors are wide (i.e., the quoted confidence intervals are large) 
due to limited information available, it is possible that a star has been 
misclassified.  This is most prevalent for sub-giants, where \citet{Huber_2016} 
note that $\approx55-70$\% of sub-giants are misclassified as dwarfs.  The 
probability of misclassification is usually quantified in the uncertainties given
for each star; most dwarfs that have a higher possibility of being sub-giants have
large confidence intervals.  Therefore, requiring low uncertainties will decrease 
the total sample size, but in practice it removes most misclassifications.  The 
situation is far more favourable for dwarfs and giants.  Only $1-4$\% of giant 
stars are misclassified as dwarfs, and about 7\% of dwarfs are misclassified as 
giants.  To summarise, the \epic\ labels with narrow confidence intervals are 
usually of high fidelity, and given that we have spectra, we can identify any
spurious misclassifications.


We sought to have a small overlap between our giant and main-sequence star training
sets.  Most of our giant training set is encapsulated within $0 < \logg < 3.5$, 
however there is a sparse sampling of stars reaching to $\logg \approx 4$.  We
required $\logg > 3.5$ for the \epic\ main-sequence/sub-giant star training set,
allowing for $\approx0.5$~dex of overlap between the two training sets.  We further
employed the following quality constraints: the upper and lower confidence intervals 
in $\teff$ must be below 150~K; the upper and lower confidence intervals in $\logg$ 
must be less than 0.15~dex; the $S/N$ of the \rave\ spectra must exceed 
30~pixel$^{-1}$; and $\teff \leqslant 6750$~K.  Unfortunately these strict constraints
removed most metal-poor stars, which we later found to cause the test labels to have
under-predicted abundances for dwarfs of low metallicity.  For this reason we relaxed
(ignored) those quality constraints for stars with $[{\rm Fe/H}] < -1$, and included
an additional 12 turn-off stars with $-1.6 \gtrsim [{\rm Fe/H}] \gtrsim -2.1$ from 
\citet{Ruchti_2011}.  After training
a model based on main-sequence and giant stars (Section \ref{sec:the-model}), we found 
we could identify misclassifications by leave-one-out cross-validation.  However, we 
chose not to do this because the number of likely misclassifications in the training
set was negligible ($\approx1$\%), and the improvement in main-sequence test set labels
was minimal.  The distilled sample of the \rave--\epic\ overlap catalog contains 595 
stars (583 of 4,611 from \epic).  The full training set for each model (see next sections) is shown
in Figure~\ref{fig:training-set-hrd}.


\subsection{A 3-label model ($\teff$, $\logg$, $\feh$) for all stars}
\label{sec:a-simple-model}


We have constructed a justified training set for stars across the main-sequence, sub-giant,
and red giant branch.  However the lack of overlap between \rave\ and other works have
resulted in a somewhat peculiar situation.  Detailed abundances are available from \apogee\
for all giant stars in our sample, but only imprecise metallicities are available from
\epic\ for stars on the main-sequence and the sub-giant branch.  Here we will construct 
a simple model for \emph{all} stars that only makes use of three labels ($\teff$, $\logg$, 
[Fe/H]), before we outline how we derive abundances for giant branch stars.  The complexity
for this model will be quadratic ($\teff^2$ is the highest term), where the vectorizer 
$\vecv(\ell_n)$ expands as,
\begin{eqnarray}\label{eq:vectorizer-three-label}
\vecv(\ell_n) \rightarrow \left[1, T_{{\rm eff},n}, \logg_n, [{\rm Fe/H}]_n, T_{{\rm eff},n}^2, \logg_n\,T_{{\rm eff},n}, \feh_n\,T_{{\rm eff},n}, \logg_n^2, \feh_n\,\logg_n, \feh_n^2\right]
\end{eqnarray}

\noindent{}such that $\vecv(\ell)$ produces the design matrix:
\begin{eqnarray}
	\vecv(\ell) \rightarrow \begin{bmatrix} \vecv(\ell_0) \\ \vdots \\ \vecv(\ell_{N-1}) \end{bmatrix} \quad .
\end{eqnarray}


We used no regularization ($\Lambda = 0$) for this model.  After training the model we
treated all \Nspectra\ spectra as test set objects.  In the left-hand panel of Figure 
\ref{fig:test-set-density} we show the effective temperature $\teff$ and surface gravity 
$\logg$ for all spectra.  The main-sequence and red giant branch are clearly visible.  
However, the details of stellar evolution are no longer present: the sub-giant branch is 
not discernible, and there are a number of systematic artefacts (over-densities) present
in label space.  These artefacts disappear when we require additional quality constraints 
(e.g., no peculiar morphological classifications), but the complexity of the 
Hertzsprung-Russell diagram is still not present.  Thus, we concluded that while this 
model could be useful for deriving stellar classifications (e.g., F2-type giant), the 
labels are too imprecise.


We chose to adopt separate models for the main-sequence and the red giant branch rather
than switch to a single model with higher complexity.  This choice allowed us to derive
stellar parameters for stars on the main-sequence and sub-giant branch, as well as 
detailed elemental abundances for red giant branch stars.  However, adopting two separate
models introduces the challenge of how to combine the results from two models, or how to
assign one star as `belonging' to a single model.  In Section~\ref{sec:joining-the-models}
we describe how we will use the 3-label model of the main-sequence and giant branch
(introduced in this section) to discriminate between results from a 3-label 
main-sequence model in Section \ref{sec:unevolved-star-model} and a 9-label giant 
star model in Section \ref{sec:evolved-star-model}.


\subsection{A 3-label model ($\teff$, $\logg$, $\feh$) for unevolved stars}
\label{sec:unevolved-star-model}


We 