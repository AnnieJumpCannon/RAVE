%% This file is part of the UNRAVE Project 
%% Copyright 2016 the authors.  All rights reserved.


%% TODO:
%% Co-authors: Look for \stub{}s. If you can expand it, do so!

\documentclass[preprint,trackchanges]{aastex}

\usepackage{amsmath}
\usepackage{bm}

% This section can be removed at submission.
% ------------------------------------------
\usepackage{color}
\usepackage{datenumber}

\newcounter{dateone}
\newcounter{datetwo}

\newcommand{\difftoday}[3]{%
  \setmydatenumber{dateone}{\the\year}{\the\month}{\the\day}%
  \setmydatenumber{datetwo}{#1}{#2}{#3}%
  \addtocounter{datetwo}{-\thedateone}%
  \the\numexpr(\thedatetwo)\relax\space days
}
% ------------------------------------------

\newcommand{\githash}{UNKNOWN}
%\input{vc.tex}

\newcommand{\project}[1]{\textsl{#1}}
\newcommand{\gaia}{\project{Gaia}}
\newcommand{\thecannon}{\project{The~Cannon}}
\newcommand{\rave}{\project{\acronym{RAVE}}}

\newcommand{\acronym}[1]{{\small{#1}}}

\newcommand{\stub}[1]{\textbf{#1}}

\newcommand{\teff}{T_{\mathrm{eff}}}
\newcommand{\logg}{\log g}

\newcommand{\Nstars}{483,330}

\newcommand{\argmin}[1]{\underset{#1}{\operatorname{argmin}}\,}


%\AuthorCallLimit=10


\begin{document}
% Remove at submission:
\slugcomment{{\color{red} \textbf{To appear on arXiv on 9 September 2016 (\difftoday{2016}{9}{7}away)}}}


\title{The \project{unRAVE} catalog}

\author{Andrew R. Casey}
\affil{Institute of Astronomy, Madingley Road, Cambridge CB3 0HA}

% Specific contributors to this work:
% There is some order already in mind, but final order will be determined by  contribution.
\author{Some combination of: Harry Enke, Gerry Gilmore, Keith Hawkins, David W. Hogg, Georges Kordopatis, Gal Matijevic, Melissa Ness, Jason Sanders, Matthias Steinmetz, Hans Walter-Rix}

% RAVE DR5 core team:
\author{Luca Casagrande, Cristina Chiappini, Andrea Kunder, Paul McMillan, Alessandro Siviero, Marica Valentini, Jennifer Wojno, Toma{\^z} Zwitter}

\author{and the \rave\ collaboration}


\begin{abstract}
The Milky Way may be our greatest laboratory for understanding galaxy formation.
This premise relies on the ability to measure [1] 3D positions and motions, and
[2] detailed chemical abundances, for large numbers of individual stars.  On 14
September 2016 the \gaia\ mission will release the former for $\approx2\times10^6$
stars.  However, a catalog of precise chemical abundances to complement that data
release is currently lacking.  The RAdial Velocity Experiment (\rave) survey has 
acquired spectra for 292,036 stars in common with the first \gaia\ data release, 
constituting the largest overlap of any stellar spectroscopic sample.  Here we 
perform an independent analysis of \rave\ spectra using an implementation of 
\thecannon\ that uses prior probabilities on atomic line formation and stellar
astrophysical parameters.  Our approach yields precise stellar labels at lower
S/N ratios than most \emph{ab initio} approaches, allowing us to deliver
improved radial velocities, effective temperature $\teff$, surface gravity $\logg$,
and chemical abundances of up to seven elements (Mg, Al, Si, Ca, Ti, Fe, Ni) 
for \Nstars\ stars.  We validate our results internally from repeat visits, 
and externally against existing high-resolution spectroscopic surveys.  The 
typical precision in chemical abundances is X.XX~dex.  We provide 
spectrophotometric distances based on our labels, improving upon the parallax
distances available for the fainter targets in the first \gaia\ data release.
\stub{A one-line science result based on RAVE/Hipparcos overlap}
\end{abstract}

\keywords{}

\section{Introduction} 
\label{sec:introduction}

The Milky Way is considered to be the best laboratory for understanding galaxy
formation and evolution.  This premise hinges on the ability to precisely measure 
the position, motion, and chemistry of (many) individual stars, and use those data 
to infer the structure, kinematics, and chemical enrichment of the Galaxy 
\citep[e.g.,][]{Deason_2011,Ness_2012,Ness_2013a,Ness_2013b,Casey_2012,Casey_2013,
Casey_2014a,Casey_2014b}.  % include references on: disk chemistry, halo structures
However, these quantities are not known for even 1\% of stars in the Milky Way.  
Stellar distances are famously imprecise \citep[e.g.,][]{van_Leeuwen_2007,
Jofre_2015,Madler_2016}, proper motions can be plagued by unquantified systematics
from the first epoch observations \citep[e.g.,][]{Casey_Schlaufman_2015}, and 
stellar spectroscopists frequently report significantly different chemical 
abundance patterns \citep{Smiljanic_2014}.  Countless long-standing astrophysics
problems could be resolved if those measurements were known with high fidelity: 
\stub{x? \citep{someone}; y; and z, to list just a few}.


\stub{Introducing Gaia; rely on ground-based spectroscopic surveys for chemistry}
% Introducing Gaia
% Gaia will produce a very clear snapshot, producing astrometry space motions
% What will Gaia provide eventually.
% In the first data release what do we get.
% While this is an order of magnitude better than Hipparcos, if we want to use these data to understand the chemical evolution of the galaxy then we need additional information.
% until abundances are available from gaia RVS, we rely on ground-based spectroscopic surveys to provide the chemistry

The first \project{Gaia} data release will include positions, proper motions, and 
parallaxes for approximately two million stars in the Tycho-2 \citep{Hog_2000} 
catalog \citep{Michalik_2015a,Michalik_2015b}.  After cross-matching all major 
stellar spectroscopic surveys\footnote{Specifically we cross-matched the Tycho-2
catalog against the \project{APOGEE} \citep{Zasowski_2013}, \project{Gaia-ESO} 
\citep{Gilmore_2012,Randich_2013}, \project{GALAH} \citep{DeSilva_2015},
\project{LAMOST} \citep{Cui_2012}, and \project{RAVE} \citep{Steinmetz_2006} 
surveys}, we found the RAdial Velocity Experiment (hereafter \rave) survey to 
have the largest overlap with the first \project{Gaia} data release: 292,036 
stars.  We then used the \project{Gaia} universe model \citep{Robin_2012} to 
estimate the precision in parallax and proper motions that will be available in 
\project{Gaia} DR1 for every star in these cross-matches.  Comparing the expected
precision to what is currently available, we further found that the \project{RAVE}
survey will benefit most from \project{Gaia} DR1.  The distances of 63\% of the 
\project{RAVE}--\project{Gaia} DR1 overlap sample (182,862 stars) are expected to
improve with the first \project{Gaia} data release, and 47\% of stars likely to
have better proper motions (137,211 stars).  This motivated us to examine what 
chemical abundance information was available for \rave, and to evaluate whether 
we could enable new chemodynamic studies by contributing to this existing set of
chemical abundances.


We briefly describe the \rave\ data in section \ref{sec:data}, before explaining
the model we employ in section \ref{sec:model}.  In section \ref{sec:validation}
we outline a number of validation experiments, including: internal sanity checks,
comparisons with literature samples, and investigations to ensure our results
are consistent with expectations from astrophysics.  We provide concluding
remarks in section \ref{sec:conclusion}, as well as a description of how to
access our results electronically.

\section{Data}
\label{sec:data}


\rave\ is a magnitude-limited stellar spectroscopic survey of the (nearby) Milky Way,
principally designed to measure radial velocities for up to $10^6$ stars.
Observations were conducted on the 1.2~m UK Schmidt telescope at the Australian 
Astronomical Observatory\footnote{Formerly the Anglo-Australian Observatory} from 
2003--2013.  The large 5.7~degree field-of-view and robotic fibre positioner (6dF)
made for very efficient observing: spectra for up to 150 targets could be 
simultaneously acquired.  When observations concluded in April 2013, a total of 
574,630 spectra had been collected of 483,330 unique objects. 


The target selection for \rave\ is based on the $I$-band apparent magnitude,
$9 < I < 12$, with a weak $J - K_s > 0.5$ cut near the disk and bulge\footnote{
However, see the works of \citet{Kunder_2016} and \citet{Kordopatis_2013} for 
subtleties of the selection function}.
The $I$ band was used because it overlaps with the wavelength range 
that \rave\ operates in:  8410--8795~\AA.  This setup includes the \ion{Ca}{2} 
near-infrared triplet.  These stellar absorption lines are dominated by pressure 
broadening, which makes them strong and therefore visible even in very low S/N data
or metal-poor stars.  This spectral region is also well-studied.  It is one of the
key setups used for the ground-based high-resolution \project{Gaia-ESO} survey
\citep{Gilmore_2012,Randich_2013}, and the \rave\ resolution and wavelength
coverage is comparable to the Radial Velocity Spectrometer on board the \project{Gaia}
space telescope \citep{Recio-Blanco_2016}.


The exposure times for \rave\ observations were optimised to obtain radial 
velocities for as many stars as possible.  Detailed chemical abundances were
always an important science goal of the survey, but this was a secondary objective.  
For this reason the distribution of S/N ratios in \rave\ spectra is considerably 
lower than other stellar spectroscopic surveys where chemical abundances are the 
primary motivation.  The \rave\ spectra have an effective resolution 
$\mathcal{R} \approx 7500$ and the distribution of S/N ratios peaks at 
$\approx$50~per~pixel$^{-1}$.  For comparison, the \project{GALAH} survey 
\citep{DeSilva_2015} --- which was specifically constructed for detailed chemical 
abundance analyses --- includes a wavelength range about 2.5 times larger at 
resolution $\mathcal{R} \approx 28,000$, and yet the \project{GALAH} project still 
targets for S/N $\gtrsim100$.


Despite the relatively low resolution and S/N of the spectra compared to other
surveys, the \rave\ data releases have demonstrated the team's enormous success 
in deriving radial velocities, stellar astrophysical parameters ($\teff$, $\logg$),
as well as individual chemical abundances \citep{Steinmetz_2006,Zwitter_2008,
Siebert_2011,Kordopatis_2013, Kunder_2016}.  In this work we make use of spectra
that has been reprocessed for the fifth \rave\ data release.  These re-processing
steps include: a detailed re-reduction of all the original data frames, with flux
variances propagated at every step; an updated continuum-normalization procedure;
as well as revised determinations of stellar radial velocities and morphological
classifications. At the end of this processing for each survey observation we are 
provided with: rest-frame wavelengths, continuum-normalized fluxes, fractional 
uncertainties in continuum-normalized flux values, as well as relevant metadata.  
We refer the reader to the official fifth data release paper of the \rave\ survey, 
as presented by \citet{Kunder_2016}, for more details of this re-processing.


Given the high-quality of the normalization performed by the \rave\ team, we chose
not to re-normalize the spectra.  Our tests demonstrated that the procedure 
outlined in \citet{Kunder_2016} is sufficient for our analysis procedure. Therefore,
there is a limited number of pre-processing steps that we conducted before our
analysis could begin.  Firstly, we calculated inverse variance arrays from the
fractional uncertainties provided, then re-sampled the flux and inverse variance
arrays onto a common rest-wavelength map for all stars.  Depending on the fibre 
used and the stellar radial velocity, the range of rest-frame wavelength values
varied for each star.  Given that fluxes were unavailable in the edge pixels for 
most stars, we excluded pixels outside of the rest wavelength range 
$8423.2\,{\rm \AA} \le \lambda \le 8777.6\,{\rm \AA}$.  This corresponds to about
30~pixels excluded on either side of the common wavelength array, leaving us with
945~pixels for science.


\section{Model}
\label{sec:model}

We chose to adopt a data-driven model --- rather than a physics-driven model ---
for this analysis.  Specifically we will use an implementation of \thecannon\
\citep{Ness_2015,Ness_2016}.  Although this choice complicated the construction 
of our model (e.g., see Section \ref{sec:the-training-set}), a data-driven model 
makes use of all available information in the spectrum, and lowers the S/N ratio 
at which systematic effects begin to dominate.  In other words, a well-constructed
data-driven model will yield more precise labels (e.g., stellar parameters and
chemical abundances) at low S/N ratios than most physics-driven models\footnote{
However, see \citet{Casey_2016a}}.  This is particularly relevant for the 
low-resolution \rave\ data analysed here, because about half of the spectra have
S/N $\lesssim 50$. 


Throughout this work we adopt the assumptions and nomenclature as set out in
\citet{Casey_2016b}.  One distinction here is that the \emph{labelled} set and
the \emph{training} set in this work will be the same set.  In \citet{Casey_2016b}
the labelled set contained two subsets: the training set and a distinct
validation set, allowing us to train on the \emph{training set} and then test
on the \emph{validation set}, a subset of the labelled set.  However in this work
the labelled set is relatively small, about an order of magnitude smaller than 
what was available in \citet{Casey_2016b}, and therefore we will use all labelled 
set stars in the training step and supplement our validation experiments with 
literature comparisons.  Thus, throughout this work we will use the terms 
\emph{labelled set} and \emph{training set} interchangeably.  Below we describe 
how we constructed the training set, before we outline the model that we adopt.


\subsection{The training set}
\label{sec:the-training-set}

We sought to construct a training set consisting of stars across the main-sequence,
sub-giant, and the red giant branch.  We require stars with precisely measured
effective temperature $\teff$, surface gravity $\logg$, and elemental abundances
of Fe, Mg, Si, Ti, Ca, Al, and Ni.  This proved to be difficult because the magnitude
range of \rave\ does not overlap substantially with high-resolution spectroscopic
surveys.  The fourth internal data release of the \project{Gaia-ESO} survey includes 
giant and main-sequence stars, but only 142 overlap with \rave.  The most recent 
\project{APOGEE} data release \citep{sloan_dr13} reports labels for stars on the
giant branch and (uncalibrated values for) the main-sequence, but our tests indicated
that the main-sequence labels suffered from unexpected systematic effects.  A flat, 
then `up-turning' main-sequence is present, and the metallicity gradient trends in 
the opposite direction with respect to $\logg$ on the main-sequence (i.e., metal-poor
stars sit above the isochrones, not below).  If we move to lower-resolution studies,
there are 2,369 stars that overlap with \project{LAMOST} --- of which 2,213 have positive
S/N ratios in the $g$-band (\texttt{snrg}).  However, similar systematic effects are 
seen on the main-sequence: the labels are expectedly less precise given the lower
resolution, no detailed abundance information is available\footnote{Abundance 
information is available for \project{LAMOST} stars from \citet{Ho_2016}, but that 
sample contains only giant stars}, and there is only sparse sampling across the 
Hertsprung-Russell diagram.  The \project{GALAH} survey has a good overlap with \rave,
but results from that sample are not yet precise enough for our purposes.


These constraints forced us to construct a heterogeneous training set.  Given previous
successes in transferring high S/N ratio labels from \project{APOGEE} \citep{Ness_2015,
Ness_2016,Ho_2016,Casey_2016b}, we chose to use the 1,355 stars in the \project{APOGEE}
---\rave\ overlap sample for giant star labels in the training set.  Of these, about
900 are giants according to \project{APOGEE}.  From this sample we selected stars to
have: metallicity determinations ([Fe/H] $> -5$); high S/N ratios in both the 
\project{APOGEE} and \rave\ spectra ($>200$ in \project{APOGEE}; $>50$ in \rave); and
we further required that the \project{ASPCAP} pipeline did not report any peculiar
flags (\texttt{ASPCAPFLAG = 0}).  These restrictions left us with 357 stars along the
giant branch, with metallicities ranging from [Fe/H] $= -1.62$ to 0.32.


Assembling a training set for the main-sequence and sub-giant branch was more complex.
% training set from K2/EPIC?!!
% Identifying obviously bad stars that have been mis-classified by K2/EPIC

% Magic wizardry for missing and imperfect labels.

% Inserting 2MASS photometry as fake pixels

% Censoring masks on abundance labels
% Priors on astrophysical parameters from isochrones

% Include velocities and rotational velocities at test time?


\section{Validation experiments}
\label{sec:validation}


\subsection{Internal validation}
We first conducted leave-one-out cross-validation to validate our results.
For each training set star we removed that star from the training set and
then re-trained the model.  Using that re-trained model, we then treated
the excluded star as a test star and compared our inferred labels to the
test set labels.  We show the results of this experiment in Figure
\ref{fig:loocv-comparison}.  The bias and root mean square (RMS) error in 
$\teff$ is XX~K and XX~K, and in $\logg$ is X.XX~dex and X.XX~dex,
respectively.  The bias in abundance labels is negligible --- between
X.XX to X.XX --- and the RMS varies from X.XX ([XX/H]) to X.XX ([XX/H]).


The \rave\ survey performed repeat observations for 23,288 stars  with time 
intervals ranging from a few hours to up to four years.  This timing was 
constructed to be quasi-logarithmic such that spectroscopic binaries could
be optimally identified. Most of the stars that were observed multiple times
were only observed twice, with thirteen visits being the maximum number 
of observations for any target.  These repeat observations allow us to 
quantify the level of (in)correctness in our formal errors.  For every star
with multiple visits we constructed a high S/N stacked spectrum for that
star, weighted by the inverse variances in the individual visit spectra,
which then served as the basis of comparison to all individual visits.  
Figure \ref{fig:formal-errors-comparison} shows the difference in labels 
between the comparison spectrum and a repeat visit, normalized by their 
formal errors summed in quadrature (e.g., $\Delta\logg/\sqrt{\sigma_{\logg,stacked}^2 + \sigma_{\logg,visit}^2}$).
If our measurements were unbiased by S/N and the errors were representative, 
these values should be normally distributed with a zero mean and unit variance.
\stub{Are they? Should we inflate our formal errors?}


% Internal validation:
% --> bootstrap resampling of the training set?

% Identify outliers:
% --> Chi-sq of the sample.
% --> Convex hull? Distance to plane?
% --> Stars with morphological classification
% --> Anything else?

\subsection{External validation}
\label{sec:external-validation}

Here we cross-match our results against literature samples to verify that
our labels are accurate.  Because this work is an independent analysis, our
first point of reference is against the official \rave\ data releases.  These
comparisons are shown in Figures \ref{fig:rave-dr4-comparison} and 
\ref{fig:rave-dr5-comparison}, where we show our labels with respect to the
results in the fourth and fifth \rave\ data releases, respectively.  In order
to provide a fair comparison we only show stars that meet a number of quality
flags in both samples. 
\stub{What selections qualify as a fair comparison in both samples?}
There is very good agreement in $\teff$, with a bias
and RMS of just XX~K and XXX~K, respectively.  We find our $\logg$ values are
closest to the `calibrated' \rave\ surface gravities, which make use of 
asteroseismic information to correct (increase) the $\logg$ values along the
giant branch.  The calibrated \rave\ surface gravities also correct the 
location of the red clump (by $\sim0.5$~dex) to be in the same position in our
results.  However, while our $\logg$ values generally agree with the calibrated
\rave\ values along the giant branch, our $\logg$ values do differ slightly 
along the main-sequence.  This is principally because the main-sequence in 
this work tends to taper down towards higher $\logg$ values at cooler 
temperatures, whereas the comparison sample tend to have a slightly flatter
main-sequence.  This difference is not likely to have a very significant 
effect on the detailed abundance or spectrophotometric distance determinations
between these studies.

\stub{What abundances are available?}
\stub{Any differences in RAVE abundances?}


% External validation: 
% --> Gaia-ESO Survey
% --> APOGEE / EPIC (whichever was not used)
% --> Literature studies presented in Kordopatis+ 2014
% --> Gold standards (incl. exoplanet host star studies): Reddy (2003, 2006), Bensby (2014), Valenti & Fischer (2005)

\subsection{Astrophysical validation}

After verifying that our stellar labels are comparable with other
high-resolution studies, here we verify that the abundance labels we find
are consistent with expectations from astrophysics.  One example is stars in
globular clusters, which show anti-correlations in light element abundances 
\citep[e.g.,][and references therein]{Da_Costa,Carretta_2009}.  In the \rave\
survey we found 70 stars with positions and radial velocities that are
consistent with being members of globular clusters.  Specifically we identified
49 stars belonging to NGC~5139 ($\omega$~Centauri), 11 members of the retrograde
globular cluster NGC~3201, and 10 members of NGC~362.  
\stub{Were any of these stars unintentionally in the training set?}
All of these systems are well-studied and exhibit abundance anti-correlations 
in Na--O and Mg--Al \citep{people}.  Other correlations are also present 
(e.g., C--N), but those relationships are less relevant for this work because 
we do not measure C and N abundances here.


We show the detailed abundances for these globular cluster members in
Figure \ref{fig:globular-cluster-abundances}.  For comparison purposes we
have included abundances from the high-resolution studies of \citep{people}.
\stub{What do we find?}

% Assuming we recover these relationships...
Indeed, this demonstrates that our abundance labels can be used to identify
globular cluster members that are now tidally disrupted \citep{Anguiano_2016,Kuzma_2016,Navin_2016},
even for stars with low S/N ratios.

% Some literature notes:
% See http://iopscience.iop.org/article/10.1088/0004-637X/731/1/64/pdf
% NGC 3201 has Na-O anti-correlation: Carretta et al. (2009)
% NGC 362 has "two discrete groups along the Na-O anti-correlation": https://arxiv.org/abs/1307.4085
% NGC 362 has Mg-Al relationship (low Mg scatter, large Al scatter) --> Carretta (2013)
% NGC 3201 has Mg-Al relationship (http://arxiv.org/pdf/1305.3645.pdf and references therein)


% Astrophysical validation:
% --> Residuals as a function of galactic position? (DIBS)



\section{Discussion}
\label{sec:discussion}





\section{Conclusion}
\label{sec:conclusion}

% We have introduced a non-parametric version of The Cannon that uses strict priors on line formation (censoring masks), and prior probability distributions on astrophysical parameters, using isochrones.
% We ran it on the entire RAVE spectra.
% We deliver effective temperature, surface gravity, and abundances of up to 8 elements for \Nstars.  Our internal and external validation tests suggest that we achieve a typical precision of X~K in $\teff$, 0.0X~dex in $\logg$, and $\approx{}X.XX$ in individual chemical abundances.  This catalog constitutes the largest collection of stellar abundances for stars in the first \project{Gaia} data release.  When combined with positions and 3D velocities from \project{Gaia}, the \project{UNRAVE} catalog will likely be crucial for understanding our local place in the Milky Way.


\subsection*{Access the results electronically}

% TODO: Replace ZENODO-URL with a data url
Source code for this project is available at \texttt{\url{https://www.github.com/AnnieJumpCannon/RAVE}}.  
This document was compiled from revision hash \texttt{\githash} in that repository.
Inferred labels and covariance matrices are available electronically with 
associated metadata at \texttt{\url{ZENODO-URL}} \citep{DATA_REPOSITORY}.  
Please note that it is a condition of using these results that the fifth \rave\ 
data release by \citet{Kunder_2016} must (also) be cited, as the work presented 
here would not have been possible without the tireless efforts of the entire 
\rave\ collaboration, past and present.


\acknowledgements
A.~R.~C. warmly thanks Jonathan Bird (Vanderbilt) and Sven Buder (MPIA).
This research made use of: Astropy, a community-developed core Python package for
Astronomy \citep{astropy}, NASA's Astrophysics Data System Bibliographic Services;
and \project{TOPCAT} \citep{Taylor_2005}.
This work was partly supported by the European Union FP7 programme through ERC 
grant number 320360.
Funding for RAVE has been provided by: the Australian Astronomical Observatory; 
the Leibniz-Institut fuer Astrophysik Potsdam (AIP); the Australian National 
University; the Australian Research Council; the French National Research Agency;
the German Research Foundation (SPP 1177 and SFB 881); the European Research 
Council (ERC-StG 240271 Galactica); the Istituto Nazionale di Astrofisica at 
Padova; The Johns Hopkins University; the National Science Foundation of the USA
(AST-0908326); the W. M. Keck foundation; the Macquarie University; the 
Netherlands Research School for Astronomy; the Natural Sciences and Engineering 
Research Council of Canada; the Slovenian Research Agency; the Swiss National 
Science Foundation; the Science \& Technology Facilities Council of the UK; 
Opticon; Strasbourg Observatory; and the Universities of Groningen, Heidelberg 
and Sydney. The RAVE web site is https://www.rave-survey.org.  

\begin{thebibliography}{dummy}

\bibitem[Anguiano et al.(2016)]{Anguiano_2016} Anguiano, B., De Silva, G.~M., Freeman, K., et al.\ 2016, \mnras, 457, 2078 

\bibitem[Astropy Collaboration et 
al.(2013)]{astropy} Astropy Collaboration, Robitaille, T.~P., Tollerud, E.~J., et al.\ 2013, Astronomy \& Astrophysics, 558, AA33

\bibitem[Casey et al.(2012)]{Casey_2012} Casey, A.~R., Keller, S.~C., \& Da Costa, G.\ 2012, \aj, 143, 88 

\bibitem[Casey et al.(2013)]{Casey_2013} Casey, A.~R., Da Costa, G., Keller, S.~C., \& Maunder, E.\ 2013, \apj, 764, 39 

\bibitem[Casey et al.(2014a)]{Casey_2014a} Casey, A.~R., Keller, S.~C., Da Costa, G., Frebel, A., \& Maunder, E.\ 2014, \apj, 784, 19 

\bibitem[Casey et al.(2014b)]{Casey_2014b} Casey, A.~R., Keller, S.~C., Alves-Brito, A., et al.\ 2014, \mnras, 443, 828 

\bibitem[Casey \& Schlaufman(2015)]{Casey_Schlaufman_2015} Casey, A.~R., \& Schlaufman, K.~C.\ 2015, \apj, 809, 110 

\bibitem[Casey(2016)]{Casey_2016a} Casey, A.~R.\ 2016, \apjs, 223, 8 

\bibitem[Casey et al.(2016)]{Casey_2016b} Casey, A.~R., Hogg, D.~W., Ness, M., et al.\ 2016, arXiv:1603.03040 

\bibitem[Cui et al.(2012)]{Cui_2012} Cui, X.-Q., Zhao, Y.-H., Chu, Y.-Q., et al.\ 2012, Research in Astronomy and Astrophysics, 12, 1197 

\bibitem[De Silva et al.(2015)]{DeSilva_2015} De Silva, G.~M., Freeman, K.~C., Bland-Hawthorn, J., et al.\ 2015, \mnras, 449, 2604 

\bibitem[Deason et al.(2011)]{Deason_2011} Deason, A.~J., Belokurov, V., \& Evans, N.~W.\ 2011, \mnras, 416, 2903 

\bibitem[Gilmore et al.(2012)]{Gilmore_2012} Gilmore, G., Randich, S., Asplund, M., et al.\ 2012, The Messenger, 147, 25

\bibitem[Ho et al.(2016)]{Ho_2016} Ho, A.~Y.~Q., Ness, M.~K., Hogg, D.~W., et al.\ 2016, arXiv:1602.00303 
 
\bibitem[H{\o}g et al.(2000)]{Hog_2000} H{\o}g, E., Fabricius, C., Makarov, V.~V., et al.\ 2000, \aap, 355, L27 

\bibitem[Jofr{\'e} et al.(2015)]{Jofre_2015} Jofr{\'e}, P., M{\"a}dler, T., Gilmore, G., et al.\ 2015, \mnras, 453, 1428 

\bibitem[Kordopatis et al.(2013)]{Kordopatis_2013} Kordopatis, G., Gilmore, G., Steinmetz, M., et al.\ 2013, \aj, 146, 134 

\bibitem[Kunder et al.(2016)]{Kunder_2016} Kunder, A., et al.\ 2016, in preparation

\bibitem[Kuzma et al.(2016)]{Kuzma_2016} Kuzma, P.~B., Da Costa, G.~S., Mackey, A.~D., \& Roderick, T.~A.\ 2016, \mnras, 461, 3639 

\bibitem[M{\"a}dler et al.(2016)]{Madler_2016} M{\"a}dler, T., Jofr{\'e}, P., Gilmore, G., et al.\ 2016, arXiv:1606.03015 

\bibitem[Michalik et al.(2015a)]{Michalik_2015a} Michalik, D., Lindegren, L., \& Hobbs, D.\ 2015, \aap, 574, A115 

\bibitem[Michalik et al.(2015b)]{Michalik_2015b} Michalik, D., Lindegren, L., Hobbs, D., \& Butkevich, A.~G.\ 2015, \aap, 583, A68 

\bibitem[Navin et al.(2016)]{Navin_2016} Navin, C.~A., Martell, S.~L., \& Zucker, D.~B.\ 2016, arXiv:1606.06430 

\bibitem[Ness et al.(2012)]{Ness_2012} Ness, M., Freeman, K., Athanassoula, E., et al.\ 2012, \apj, 756, 22 

\bibitem[Ness et al.(2013a)]{Ness_2013a} Ness, M., Freeman, K., Athanassoula, E., et al.\ 2013, \mnras, 430, 836 

\bibitem[Ness et al.(2013b)]{Ness_2013b} Ness, M., Freeman, K., Athanassoula, E., et al.\ 2013, \mnras, 432, 2092 

\bibitem[Ness et al.(2015)]{Ness_2015} Ness, M., Hogg, D.~W., Rix, H.-W., Ho, A.~Y.~Q., \& Zasowski, G.\ 2015, \apj, 808, 16 

\bibitem[Ness et al.(2016)]{Ness_2016} Ness, M., Hogg, D.~W., Rix, H.-W., et al.\ 2016, \apj, 823, 114 

\bibitem[Randich et al.(2013)]{Randich_2013} Randich, S., Gilmore, G., \& Gaia-ESO Consortium 2013, The Messenger, 154, 47 

\bibitem[Recio-Blanco et al.(2016)]{Recio-Blanco_2016} Recio-Blanco, A., de Laverny, P., Allende Prieto, C., et al.\ 2016, \aap, 585, A93 

\bibitem[Robin et al.(2012)]{Robin_2012} Robin, A.~C., Luri, X., Reyl{\'e}, C., et al.\ 2012, \aap, 543, A100 

\bibitem[Siebert et al.(2011)]{Siebert_2011} Siebert, A., Williams, M.~E.~K., Siviero, A., et al.\ 2011, \aj, 141, 187 

\bibitem[SDSS Collaboration et al.(2016)]{sloan_dr13} SDSS Collaboration, Albareti, F.~D., Allende Prieto, C., et al.\ 2016, arXiv:1608.02013 

\bibitem[Smiljanic et al.(2014)]{Smiljanic_2014} Smiljanic, R., Korn, A.~J., Bergemann, M., et al.\ 2014, \aap, 570, A122 

\bibitem[Steinmetz et al.(2006)]{Steinmetz_2006} Steinmetz, M., Zwitter, T., Siebert, A., et al.\ 2006, \aj, 132, 1645 

\bibitem[Taylor(2005)]{Taylor_2005} Taylor, M.~B.\ 2005, Astronomical Data Analysis Software and Systems XIV, 347, 29 

\bibitem[van Leeuwen(2007)]{van_Leeuwen_2007} van Leeuwen, F.\ 2007, \aap, 474, 653 

\bibitem[Zasowski et al.(2013)]{Zasowski_2013} Zasowski, G., Johnson, J.~A., Frinchaboy, P.~M., et al.\ 2013, \aj, 146, 81 

\bibitem[Zwitter et al.(2008)]{Zwitter_2008} Zwitter, T., Siebert, A., Munari, U., et al.\ 2008, \aj, 136, 421 

\end{thebibliography}

\clearpage

\begin{figure}[p]
\caption{A H-R diagram showing the training set labels.\label{fig:training-set-hrd}}
\end{figure}

\begin{figure}[p]
\caption{A H-R diagram showing the test set labels.\label{fig:test-set-hrd}}
\end{figure}

\begin{figure*}[p]
\caption{The RMS of the test set labels as a function of S/N ratio for repeated stars in the test set.\label{fig:test-set-repeats}}
\end{figure*}

\begin{figure*}[p]
\caption{Stellar label comparison between the \project{UNRAVE} catalog and the \project{RAVE} fourth data release \citep{Kordopatis_2014}.\label{fig:rave-dr4-comparison}}
\end{figure*}


\begin{figure*}[p]
\caption{Stellar label comparison between the \project{UNRAVE} catalog and the \project{RAVE} fifth data release \citep{Kunder_2016}.\label{fig:rave-dr5-comparison}}
\end{figure*}

\begin{figure*}[p]
\caption{Stellar label comparison between the \project{UNRAVE} catalog and the \project{Gaia-ESO Survey} fourth internal data release.\label{fig:ges-dr4-comparison}}
\end{figure*}

\begin{figure*}[p]
\caption{Stellar label comparisons between the \project{UNRAVE} catalog and the samples discussed in \citep{Kordopatis_2014}}
\end{figure*}

\end{document}
