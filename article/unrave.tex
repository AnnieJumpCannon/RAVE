%% This file is part of the UNRAVE Project 
%% Copyright 2016 the authors.  All rights reserved.

\documentclass[preprint2,trackchanges]{aastex}

\usepackage{amsmath,amssymb}
\usepackage{bm}

\newcommand{\project}[1]{\textsl{#1}}
\newcommand{\thecannon}{\project{The~Cannon}}
\newcommand{\acronym}[1]{{\small{#1}}}
\newcommand{\rave}{\project{\acronym{RAVE}}}
\newcommand{\logg}{\log g}
\newcommand{\Teff}{T_{\mathrm{eff}}}
\newcommand{\argmin}[1]{\underset{#1}{\operatorname{argmin}}\,}

%\AuthorCallLimit=10

\begin{document}

\title{The \project{UNRAVE} catalog of stellar astrophysical parameters and chemical abundances}

\author{Andrew R. Casey}
\affil{Institute of Astronomy, Madingley Road, Cambridge CB3 0HA}

\author{Some combination of: Hawkins, Ness, Hogg, HWR, Gilmore, Matijevic, RAVE collaboration}
%\altaffiltext{3}{\texttt{arc@ast.cam.ac.uk}}

\begin{abstract}
% Gaia will release stuff
% Much we want to do will involve chemistry + kinematics
% RAVE has the largest overlap with Gaia DR1, but chemistry is unavailable for most stars
% We perform an independent analysis of RAVE spectra w/ The Cannon, censoring, + upgrades, to derive astrophysical parameters and detailed chemical abundances.
% We produce a catalog of up to N abundances for X stars, constituting the largest
% catalog of chemical abundances that overlap with Gaia by a factor of X.

\end{abstract}

\keywords{}

\section{Introduction} 
\label{sec:introduction}


\section{Methods}
\label{sec:methods}


\section{Validation}
\label{sec:validation}


\section{Conclusion}
\label{sec:conclusion}


\acknowledgements
% We thank,......

This research made use of: Astropy, a community-developed core Python package for
Astronomy \citep{astropy}, NASA's Astrophysics Data System Bibliographic Services;
 and \project{TOPCAT} \citep{Taylor2005}.
This work was partly supported by the European Union FP7 programme through ERC 
grant number 320360.
Funding for RAVE has been provided by: the Australian Astronomical Observatory; the Leibniz-Institut fuer Astrophysik Potsdam (AIP); the Australian National University; the Australian Research Council; the French National Research Agency; the German Research Foundation (SPP 1177 and SFB 881); the European Research Council (ERC-StG 240271 Galactica); the Istituto Nazionale di Astrofisica at Padova; The Johns Hopkins University; the National Science Foundation of the USA (AST-0908326); the W. M. Keck foundation; the Macquarie University; the Netherlands Research School for Astronomy; the Natural Sciences and Engineering Research Council of Canada; the Slovenian Research Agency; the Swiss National Science Foundation; the Science \& Technology Facilities Council of the UK; Opticon; Strasbourg Observatory; and the Universities of Groningen, Heidelberg and Sydney. The RAVE web site is at https://www.rave-survey.org.  



\begin{thebibliography}{dummy}
\bibitem[Astropy Collaboration et 
al.(2013)]{astropy} Astropy Collaboration, Robitaille, T.~P., Tollerud, E.~J., et al.\ 2013, Astronomy \& Astrophysics, 558, AA33

\bibitem[Taylor(2005)]{Taylor2005} Taylor, M.~B.\ 2005, Astronomical Data Analysis Software and Systems XIV, 347, 29 

\end{thebibliography}

\clearpage

%\begin{figure}[p]
%\caption{Choose two wavelengths (one continuum and one interesting)
%  and plot some scatter plots of flux vs various parameters, and also
%  cross-validation results for the
%  hyper-parameters.\label{fig:onewavelength}}
%\end{figure}


\end{document}
