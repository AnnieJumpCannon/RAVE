%% This file is part of the UNRAVE Project 
%% Copyright 2016 the authors.  All rights reserved.

\documentclass[preprint2,trackchanges]{aastex}

\usepackage{amsmath,amssymb}
\usepackage{bm}

\newcommand{\project}[1]{\textsl{#1}}
\newcommand{\thecannon}{\project{The~Cannon}}
\newcommand{\acronym}[1]{{\small{#1}}}
\newcommand{\rave}{\project{\acronym{RAVE}}}
\newcommand{\logg}{\log g}
\newcommand{\teff}{T_{\mathrm{eff}}}
\newcommand{\argmin}[1]{\underset{#1}{\operatorname{argmin}}\,}
\newcommand{\stub}[1]{\textbf{#1}}
\newcommand{\Nstars}{483,330}

%\AuthorCallLimit=10

\begin{document}

\title{The \project{UNRAVE} catalog}

\author{Andrew R. Casey}
\affil{Institute of Astronomy, Madingley Road, Cambridge CB3 0HA}

\author{Some combination of: Harry Enke, Gerry Gilmore, Keith Hawkins, David W. Hogg, Gal Matijevic, Melissa Ness, Jason Sanders, Matthias Steinmetz, Hans Walter-Rix, and any other \project{RAVE} people who have contributed to the data analysis}
%\altaffiltext{3}{\texttt{arc@ast.cam.ac.uk}}

\author{and the \project{RAVE} collaboration}

\begin{abstract}
On 14 September 2016 the \project{Gaia} mission will release positions, proper motions, and parallaxes for about two million stars.  Most studies intending to make use of those data would benefit enormously from stellar chemical abundances.  The RAdial Velocity Experiment (\project{RAVE}) survey has acquired spectra for approximately 290,000 stars in common with the first \project{Gaia} data release, constituting the largest overlap of any stellar spectroscopic sample.  Here we perform an independent analysis of the \project{RAVE} spectra using a new non-parametric implementation of \thecannon\ that employs prior probabilities on atomic line formation and stellar astrophysical parameters.  We deliver effective temperature $\teff$, surface gravity $\logg$, and chemical abundances of up to seven elements (Mg, Al, Si, Ca, Ti, Fe, Ni) for \Nstars\ stars.  We validate our results internally from repeat visits, and externally against existing high-resolution spectroscopic surveys.  The typical precision in chemical abundances is $X.XX$~dex.  
\end{abstract}

\keywords{}

\section{Introduction} 
\label{sec:introduction}

\stub{Chemistry and kinematics are essential for unfolding the Galaxy, etc.}

\stub{
Gaia will do this, and until Gaia releases abundances for stars from RVS 
spectra, we rely on ground-based surveys to provide chemical abundances.
This includes GES, APOGEE, etc.
}

The first \project{Gaia} data release includes positions, proper motions, and 
parallaxes for approximately two million stars in the Tycho-2 catalog \citep{tycho2}.  
After cross-matching all major stellar spectroscopic surveys\footnote{Specifically 
we cross-matched the Tycho-2 catalog against {APOGEE} \citep{apogee}, 
\project{GALAH} \citep{galah}, \project{Gaia-ESO} \citep{ges}, \project{LAMOST}
\citep{lamost}, and \project{RAVE} \citep{rave}}, we found the RAdial Velocity
Experiment (hereafter \rave) 
survey to have the largest overlap with the first \project{Gaia} data release: 
292,036 stars.  We then used the \project{Gaia} universe model \citep{GUMS} to 
estimate the precision in parallax and proper motions that will be available in 
\project{Gaia} DR1 for every star in these cross-matches.  Comparing the 
expected precision to what is currently available, we further found that the 
\project{RAVE} survey will benefit most from \project{Gaia} DR1.  The distances
of 63\% of the \project{RAVE}--\project{Gaia} DR1 overlap sample (182,862 stars)
are expected to improve with the first \project{Gaia} data release, and 47\% of
stars likely to have better proper motions (137,211 stars).  This motivated us to
examine what chemical abundance information is currently available for \rave, and
to evaluate whether we could contribute to this existing set.

We briefly describe the \rave\ data in Section \ref{sec:data}, before explaining
the model we employ in Section \ref{sec:model}.  In Section \ref{sec:validation}
we outline a number of validation experiments, including: internal sanity checks,
comparisons with literature samples, and investigations to ensure our results
are consistent with expectations from astrophysics.  We provide concluding
remarks in Section \ref{sec:conclusion}, as well as a description of how to
access the data.


\section{Data}
\label{sec:data}

The \rave\ survey started observations using the 6dF system on the 1.2~m UK 
Schmidt telescope at the Australian Astronomical Observatory\footnote{Formerly 
the Anglo-Australian Observatory}.  The large 5.7~degree field-of-view and 
robotic fibre positioner made for very efficient observing: up to 150 targets 
could be simultaneously obtained.  When observations concluded in April 2013, a 
total of 574,630 spectra had been collected of 483,330 unique objects.

RAVE is a magnitude-limited survey, targeting stars with apparent $I$-band
magnitudes between $9 < I < 13$.  The $I$ band was used as the selection
magnitude because it overlaps with the wavelength range that \rave\ operates in:
8410--8795~\AA.  This setup includes the \ion{Ca}{2} near-infrared triplet,
strong stellar absorption lines that are visible even in very low S/N data 
(i.e., they are very useful for radial velocity determination).  


% When did RAVE observe?
% Where from

% target selection.
% wavelength range


% How many data releases
% What was in previous data releases?


% Data reduction
% Normalization
% Flux errors

As the name implies, the \rave\ survey was principally designed to obtain radial
velocity measurements for up to $10^6$ stars.  Detailed chemical abundances were
a secondary science goal of the survey.  For this reason the distribution
of S/N ratios for \rave\ spectra is considerably lower than stellar 
spectroscopic surveys where chemical abundances are the defining objective.  The 
\rave\ spectra cover $\approx38$~nm in the infrared with an effective resolution
$\mathcal{R} \approx 7500$, and the S/N peaks at $\approx$50~per~pixel$^{-1}$.
For comparison, the \project{GALAH} survey \citep{DaSilva_2015} --- which was 
specifically constructed for detailed chemical abundance analyses --- includes a
wavelength range about 2.5 times larger at resolution $\mathcal{R} \approx 28,000$,
and target for S/N $\gtrsim100$.


Despite the relatively low resolution and S/N of the spectra, the \rave\ data
releases have demonstrated the team's enormous success in deriving radial 
velocities, stellar astrophysical parameters ($\teff$, $\logg$), as well as 
individual chemical abundances \citep{Steinmetz_2006,Zwitter_2008,Siebert_2011,
Kordopatis_2014, Kunder_2016}.  The data used for this analysis are the result
of a thorough reprocessing of all survey data.  This includes a detailed
re-reduction of the original data frames, an updated continuum-normalization
procedure, as well as an improvement on the quantification of flux errors in
each pixel.  At the end of this processing for each survey observation we are 
provided with: wavelengths, continuum-normalized fluxes, as well as 
$1\sigma$ uncertainties in flux values.  We refer the reader to the official 
fifth data release paper of the \rave\ survey, as presented by
\citet{Kunder_2016}, for more details of this reprocessing.


We chose not to re-normalize the spectra, as our tests proved to us that the
normalization procedure outlined in \citet{Kunder_2016} is sufficient for our
analysis.  We re-sampled all spectra onto a common rest-wavelength map.
Depending on the fibre used and stellar radial velocity, the rest-frame wavelength
range varied for each star.  Given that fluxes were unavailable in the edge
pixels for most stars, we excluded pixels outside of the rest wavelength range
$8423.2{\rm \AA} < \lambda < 8777.6{\rm \AA}$.


\section{Model}
\label{sec:model}

Given the relatively low S/N ratios in the \rave\ spectra, we chose to adopt a 
data-driven model (rather than a physics-driven model) for this analysis.




% Describe The Cannon briefly
% training set from APOGEE

% Initial tests using a single model for all RAVE spectra
% Censoring masks
% Priors on astrophysical parameters from isochrones

% Include velocities at test time?
% Train on fibre number? Fit for spectral resolution at test time?


\section{Validation}
\label{sec:validation}

% Chi-sq of the sample.
% Identify outliers.
% Convex hull?

% Stars with morphological classification

% Internal validation:
% --> repeat visits: are our formal errors reasonable?
% --> bootstrap resampling of the training set?

% External validation: 
% --> RAVE Data Release 4
% --> Gaia-ESO Survey
% --> Asteroseismic nu_max and delta_nu available anywhere?
% --> Literature studies presented in Kordopatis+ 2014

% Astrophysical validation:
% --> Identify open clusters and look at their metallicity spreads
% --> Identify globular clusters and look to see whether we capture "expected" abundance trends/spreads that are seen elsewhere in the literature
% --> Residuals as a function of galactic position? (DIBS)

\section{Conclusion}
\label{sec:conclusion}

% We have introduced a non-parametric version of The Cannon that uses strict priors on line formation (censoring masks), and prior probability distributions on astrophysical parameters, using isochrones.
% We ran it on the entire RAVE spectra.
% We deliver effective temperature, surface gravity, and abundances of up to 8 elements for \Nstars.  Our internal and external validation tests suggest that we achieve a typical precision of X~K in $\teff$, 0.0X~dex in $\logg$, and $\approx{}X.XX$ in individual chemical abundances.  This catalog constitutes the largest collection of stellar abundances for stars in the first \project{Gaia} data release.  When combined with positions and 3D velocities from \project{Gaia}, the \project{UNRAVE} catalog will likely be crucial for understanding our local place in the Milky Way.

\acknowledgements
We thank Sven Buder (MPIA), others??.
This research made use of: Astropy, a community-developed core Python package for
Astronomy \citep{astropy}, NASA's Astrophysics Data System Bibliographic Services;
 and \project{TOPCAT} \citep{Taylor2005}.
This work was partly supported by the European Union FP7 programme through ERC 
grant number 320360.
Funding for RAVE has been provided by: the Australian Astronomical Observatory; the Leibniz-Institut fuer Astrophysik Potsdam (AIP); the Australian National University; the Australian Research Council; the French National Research Agency; the German Research Foundation (SPP 1177 and SFB 881); the European Research Council (ERC-StG 240271 Galactica); the Istituto Nazionale di Astrofisica at Padova; The Johns Hopkins University; the National Science Foundation of the USA (AST-0908326); the W. M. Keck foundation; the Macquarie University; the Netherlands Research School for Astronomy; the Natural Sciences and Engineering Research Council of Canada; the Slovenian Research Agency; the Swiss National Science Foundation; the Science \& Technology Facilities Council of the UK; Opticon; Strasbourg Observatory; and the Universities of Groningen, Heidelberg and Sydney. The RAVE web site is https://www.rave-survey.org.  



\begin{thebibliography}{dummy}
\bibitem[Astropy Collaboration et 
al.(2013)]{astropy} Astropy Collaboration, Robitaille, T.~P., Tollerud, E.~J., et al.\ 2013, Astronomy \& Astrophysics, 558, AA33

\bibitem[Taylor(2005)]{Taylor2005} Taylor, M.~B.\ 2005, Astronomical Data Analysis Software and Systems XIV, 347, 29 

\end{thebibliography}

\clearpage

\begin{figure}[p]
\caption{A H-R diagram showing the training set labels.\label{fig:training-set-hrd}}
\end{figure}

\begin{figure}[p]
\caption{A H-R diagram showing the test set labels.\label{fig:test-set-hrd}}
\end{figure}

\begin{figure}[p]
\caption{The RMS of the test set labels as a function of S/N ratio for repeated stars in the test set.\label{fig:test-set-repeats}}
\end{figure}

\begin{figure}[p]
\caption{Stellar label comparison between the \project{UNRAVE} catalog and the \project{RAVE} fourth data release.\label{fig:dr4-comparison}}
\end{figure}

\begin{figure}[p]
\caption{Stellar label comparison between the \project{UNRAVE} catalog and the \project{Gaia-ESO Survey} fourth internal data release.\label{fig:dr4-comparison}}
\end{figure}

\begin{figure}[p]
\caption{Stellar label comparisons between the \project{UNRAVE} catalog and the samples discussed in \citep{kordopatis_2014}}
\end{figure}

\end{document}
